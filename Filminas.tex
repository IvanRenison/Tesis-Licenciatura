\documentclass[spanish, aspectratio=169, hidecontrols]{beamer}
%\usetheme{moloch}

\useoutertheme{miniframes}
\usepackage{etoolbox}
\makeatletter
\patchcmd{\slideentry}{\advance\beamer@tempdim by -.05cm}{\advance\beamer@tempdim by\beamer@vboxoffset\advance\beamer@tempdim by\beamer@boxsize\advance\beamer@tempdim by 1.2\pgflinewidth}{}{}
\patchcmd{\slideentry}{\kern\beamer@tempdim}{\advance\beamer@tempdim by 5pt\advance\beamer@tempdim by\wd\beamer@sectionbox\kern\beamer@tempdim}{}{}
\makeatother

\addtobeamertemplate{headline}{}{\vspace*{-5mm}}

\setbeamertemplate{navigation symbols}{}
%\setbeamertemplate{footline}[frame number]

\usepackage{fontspec}

\usefonttheme{serif}

\usepackage{polyglossia}
\setdefaultlanguage{spanish}

\usepackage[math-style=ISO,bold-style=ISO]{unicode-math}

\usepackage{minted}
\setminted{autogobble}

\usepackage{relsize}
\usepackage{xspace}

\usepackage{pgfplots} % Para los gráficos
\pgfplotsset{compat=1.18}

%\usepackage{tikz}
%\usepackage{pgf} % Para dibujar la cuadrícula
%% Definir un fondo con cuadrícula
%\usebackgroundtemplate{
%  \begin{tikzpicture}
%    \draw[step=0.5cm,gray,very thin] (0,0) grid (\paperwidth,\paperheight);
%  \end{tikzpicture}
%}


\newcommand\cpp{C\nolinebreak[4]\hspace{-.05em}\raisebox{.4ex}{\relsize{-3}{\textbf{++}}}\xspace}


\title{Implementación de bases de Gröbner no conmutativas en C++ con un poquito de paralelismo}
\author{Iván Renison}
\institute{Facultad de Matemática, Astronomía, Física y Computación\\
  Universidad Nacional de Córdoba}
\date{2025-03-06}

% \pgfdeclareimage[height=0.5cm]{university-logo}{logos/unc.jpg}
% \logo{\pgfuseimage{university-logo}}


\begin{document}

% Cosas que tengo que tener listas:
% El repositorio de la librería abierto en el navegador y en vscode
% La terminal abierta en el directorio de la librería

\begin{frame}[plain]
  \title{Implementación de bases de Gröbner no conmutativas en \cpp con un poquito de paralelismo}
  \titlepage
\end{frame}

%\begin{frame}[plain]
%  \frametitle{Índice}
%  \tableofcontents
%\end{frame}

\section{Preliminares}

\begin{frame}
  \frametitle{Polinomios}

  \pause

  \begin{exampleblock}{Algunos polinomios:}
    \begin{itemize}
      \item $3 + x - \frac{1}{2} x^2$
      \item $5 + 3 y - 2 x y + x^3 y^5$
    \end{itemize}
  \end{exampleblock}

  \pause

  Fijemos:
  \begin{itemize}
    \item Un alfabeto $X$
    \item Un cuerpo $K$
  \end{itemize}

  El conjunto de polinomios se denota $K[X]$.

  \pause

  \begin{exampleblock}{En los ejemplos, $K = {\color{blue}ℚ}$ y $X = {\color{red}\{x, y\}}$:}
    \begin{itemize}
      \item ${\color{blue}3} + {\color{blue}1} {\color{red}x} + {\color{blue}(-\frac{1}{2})} {\color{red}{x}}^2$
      \item ${\color{blue}5} + {\color{blue}3} {\color{red}y} + {\color{blue}(-2)} {\color{red}x} {\color{red}y} + {\color{blue}1} {\color{red}{x}}^3 {\color{red}{y}}^5$
    \end{itemize}
  \end{exampleblock}

\end{frame}

\begin{frame}
  \frametitle{¿De qué están hechos los polinomios?}
  \pause
  Son combinaciones lineales con coeficientes en $K$ de monomios.
  \begin{exampleblock}{}
    {\color{blue}Coeficientes} y {\color{violet}monomios}:
    \begin{itemize}
      \item ${\color{blue}3} + {\color{blue}1} {\color{violet}x} + {\color{blue}(-\frac{1}{2})} {\color{violet}x^2}$
      \item ${\color{blue}5} + {\color{blue}3} {\color{violet}y} + {\color{blue}(-2)} {\color{violet}x y} + {\color{blue}1} {\color{violet}x^3 y^5}$
    \end{itemize}
  \end{exampleblock}
\end{frame}

\begin{frame}
  \frametitle{Monomios}
  \pause
  El producto entre las variables es conmutativo.
  \pause
  \begin{exampleblock}{}
    \begin{itemize}
      \item $xy = yx$
      \item $x^3 y^5 = xyyxyxyy$
    \end{itemize}
  \end{exampleblock}

\end{frame}

\begin{frame}
  \frametitle{Monomios no conmutativos}
  \pause
  Se pueden considerar otros monomios en los que el producto entre las variables no conmuta.
  \begin{exampleblock}{}
    \begin{itemize}
      \item $xy ≠ yx$
      \item $x^3 y^5 ≠ xyyxyxyy$
      \item $casa ≠ asac$
    \end{itemize}
  \end{exampleblock}
\end{frame}

\begin{frame}
  \frametitle{Polinomios no conmutativos}
  Son combinaciones lineales con coeficientes en K de monomios no conmutativos
  \pause
  \begin{exampleblock}{Por ejemplo en los racionales:}
    \begin{columns}
      \begin{column}{0.5\textwidth}
        \begin{itemize}
          \item $f_0 = x$
          \item $f_1 = xy + yz$
          \item $f_2 = 3 xyy - 2 xzxy + \frac{4}{3} yzzx$
        \end{itemize}
      \end{column}
      \begin{column}{0.5\textwidth}
        \pause
        Se pueden sumar:
        \begin{itemize}
          \item $f_0 + f_1 = f_1 + f_0 = x + xy + yz$
          \item $f_1 + (\frac{1}{2}xy + 2 xz) = \frac{3}{2} xy + 2 xz + yz$
        \end{itemize}
        \pause
        Y se pueden multiplicar:
        \begin{itemize}
          \item $f_0 f_1 = xxy + xyz$
          \item $f_1 f_0 = xyx + yzx$
        \end{itemize}
      \end{column}
    \end{columns}
  \end{exampleblock}

  \pause

  Se denota con $K⟨X⟩$ al conjunto de polinomios no conmutativos
\end{frame}

% Definir ideal en general y dar un ejemplo para el caso conmutativo

% \begin{frame}
%   \frametitle{Ordenes monomiales}
%   $≤_{deglex}$ ordena a $⟨X⟩$ primero por largo y después por orden lexicográfico.
%   \pause
%   \begin{exampleblock}{En los del ejemplo de antes:}
%     \begin{itemize}
%       \item $f_0 = x$
%       \item $f_1 = xy + yz$
%       \item $f_2 = 3 xyy - 2 xzxy + \frac{4}{3} yzzx$
%     \end{itemize}
%     \pause
%     Tenemos:
%     \[x ≤_{deglex} xy ≤_{deglex} yz ≤_{deglex} xyy ≤_{deglex} xzxy ≤_{deglex} yzzx\]
%   \end{exampleblock}
% \end{frame}

\begin{frame}
  \frametitle{Problema principal del trabajo}
  Dados $G ⊆ K⟨X⟩$ y $f ∈ K⟨X⟩$, determinar si existen $g_1, …, g_n ∈ G$ y $c_1, …, c_n, c'_1, …, c'_n ∈ K⟨X⟩$ tales que
  \[ f = ∑_{i = 1}^n c_i g_i c'_i \text{.}\]
  \pause
  \begin{exampleblock}{Por ejemplo, si:}
    \begin{itemize}
      \item $g_0 = xy + yz$
      \item $g_1 = yx + zx$
      \item $G = \{g_0, g_1\}$
    \end{itemize}
    Entonces:
    \begin{itemize}
      \item Para $f_4 = xyx - yyx$ si vale\pause, porque es igual a $g_0 x - y g_1$. % Usar f y f'
      \pause
      \item ¿Y para $f_5 = xyz + zyx$ valdrá? \pause Lo veremos.
    \end{itemize}
  \end{exampleblock}

\end{frame}

\begin{frame}
  \frametitle{¿Qué aplicaciones practicas tiene?}

  En identidades sobre un operador asociativo.
  \begin{exampleblock}{Por ejemplo:}
    Si tenemos un producto asociativo, constantes $a, b, c, d$ y axiomas:
    \begin{itemize}
      \item $aba = a$
      \item $bab = b$
      \item $dc = ab$
      \item $cd = ba$
    \end{itemize}
    Nos podemos preguntar si otras igualdades se pueden deducir a partir de estas.
  \end{exampleblock}
\end{frame}

\begin{frame}
  \frametitle{Ideales}
  \pause
  Sea $G ⊆ K⟨X⟩$, se define
  \[ (G) = \{∑_{i = 1}^n c_i g_i c_i' : g_1, …, g_n ∈ G, c_1, …, c_n, c_1', …, c_n' ∈ K⟨X⟩\} \]
  A $(G)$ se lo llama el ideal generado por $G$.
\end{frame}

\begin{frame}
  \frametitle{Problema prinicpal reformualdo}
  Dados $G ⊆ K⟨X⟩$ y $f ∈ K⟨X⟩$, determinar si vale que
  \[ f ∈ (G) \text{.} \]
  \pause
  El problema principal también se llama problema de pertenencia al ideal.
\end{frame}

\begin{frame}
  \frametitle{Decidibilidad}
  \pause
  Este problema es semi-decidible\pause, pero no es decidible en general.
  % Capás tengo que escribir (además de decir) porqué es semi-decidible
\end{frame}

\begin{frame}
  \frametitle{Cómo se intenta resolver}
  \begin{itemize}
    \item Hay algunos conjuntos $G$ para los cuales hay un algoritmo fácil. \pause \\
    Esos conjuntos se llaman bases de Gröbner.
    \pause
    \item Todos los $G$ tienen una base de Gröbner que genera el mismo ideal.
    \pause
    \item Pero no siempre es finita.
    \pause
    \item Cuando es finita se puede calcular. \pause Y usar el algoritmo fácil \\ para resolver el problema.
  \end{itemize}
\end{frame}

% \begin{frame}
%   \frametitle{Teorema}
%   \pause
%   Sean $G ⊆ K⟨X⟩$, $f ∈ K⟨X⟩$, $g ∈ G$ y $c, c ∈ K⟨X⟩$. Entonces
%   \[ f ∈ (G) ⇒ f + c g c' ∈ (G) \]
% \end{frame}


% \begin{frame}
%   \frametitle{Reducciones}
%   \begin{example}
%     \begin{itemize}
%       \item $g_0 = f_1 = xy + yz$
%       \item $g_1 = f_3 = yx + zx$
%       \item $G = \{g_0, g_1\}$
%       \\
%       \item $f_4 = xyx - yyx$
%       \item $f_5 = xyz + zyx$
%       \item $f_6 = -xzx + yzx - 2xyxz - 2 xzxz$
%     \end{itemize}
%   \end{example}
% \end{frame}

\begin{frame}
  \frametitle{Algoritmos}
  \pause

  Van enumerando la base de Gröbner.

  \pause
  Hay dos algoritmos principales:
  \pause
  \begin{itemize}
    \item Buchberger: \pause va agregando los elementos de a uno.
    \item F4: \pause usa álgebra lineal para agregar los elementos de a muchos.
  \end{itemize}
\end{frame}

\section{Mi librería}

\begin{frame}
  \frametitle{Mi librería}
  Librería \texttt{ncgb} de \cpp.
  \pause
  \begin{itemize}
    \item Implementa estructuras para monomios y polinomios no conmutativos con sus operaciones básicas, como \texttt{*}, \texttt{+}, \texttt{-}, \texttt{==}, \texttt{+=}, etc.
    \pause
    \item Implementa Buchberger y F4.
    \pause
    \item Es \alert{paramétrica en el cuerpo}.
    \pause
    \item Para las matrices de un cuerpo arbitrario usa una implementación propia.
    \pause
    \item Para los racionales usa la libraría FLINT para las matrices. % Decir esto más corto
    \pause
    \item Falta implementar la optimización de Faugère-Lachartre en F4.
    \pause
    \item Incluye \alert{representación de cofactores} en Buchberger (``\textit{reconstrucción de la respuesta}'').
    \pause
    \item Tiene un poquito de paralelismo. % Poner en filmina a parte
  \end{itemize}
\end{frame}

\begin{frame}
  \frametitle{Ejemplo de mi librería}
  \begin{exampleblock}{}
    Teníamos:
    \begin{itemize}
      \item $g_0 = f_1 = xy + yz$
      \item $g_1 = f_3 = yx + zx$
      \item $G = \{g_0, g_1\}$
    \end{itemize}

    Dijimos que:
    \begin{itemize}
      \item Para $f_4 = xyx - yyx$ si vale.
      \item Para $f_5 = xyz + zyx$ no sabíamos si vale o no.
    \end{itemize}
  \end{exampleblock}
\end{frame}

\section{Trabajos futuros}

\begin{frame}
  \frametitle{Repaso}
  Implementé una librería de \cpp para calcular bases de Gröbner no conmutativas y resolver el problema de pertenencia al ideal.
\end{frame}

\begin{frame}
  \frametitle{Trabajos futuros}
  \begin{itemize}
    \item Implementar la optimización de Faugère-Lachartre en F4.
    \item Y si es posible paralelizarla.
  \end{itemize}
\end{frame}

\appendix

\begin{frame}[noframenumbering, plain]
  \frametitle{Fin}
  \begin{center}
    \Huge Gracias
  \end{center}
\end{frame}

\appendix

\section{Implementaciones previas}

\begin{frame}
  \frametitle{Implementaciones previas}
  \pause
  De Buchberger:
  \begin{itemize}
    \item Paquete GBNP de GAP.
    \item Implementación dentro de Singular.
    \item Implementación dentro de NCAlgebra.
    \item Paquete \texttt{OperatorGB} de Mathematica.
  \end{itemize}
  \pause
  De F4:
  \begin{itemize}
    \item Implementación dentro de Magma.
    \item Paquete \texttt{operator\_gb} de Python y SageMath.
  \end{itemize}
\end{frame}

\section{Comparación}

\begin{frame}
  \frametitle{Comparación}
  Algunos casos de testeo:
  \begin{align*}
    \text{FK4} =& \{a^2,\ b^2,\ c^2,\ d^2,\ e^2,\ f^2,\ ac + ba + cb,\ ae + da + ed,\ bf + db + fd,\ cf + ec + fe,\ ab + bc + ca,\ ad + de + ea, \\
      & bd + df + fb,\ ce + ef + fc,\ cd + dc,\ be + eb,\ af + fa\} \\
    \text{tri1} =& \{a^2 - 1,\ b^3 - 1,\ {(ababab^2ab^2)}^2 - 1\} \\
    \text{tri2} =& \{a^2 - 1,\ b^3 - 1,\ {(ababab^2)}^3 - 1\} \\
    \text{tri3} =& \{a^3 - 1,\ b^3 - 1,\ {(abab^2)}^2 - 1\} \\
    \text{trit4} =& \{a^3 - 1,\ b^3 - 1,\ c^4 - 1,\ {(ab)}^2 - 1,\ {(ac)}^2 - 1,\ {(bc)}^2 - 1\} \\
    \text{trit5} =& \{a^3 - 1,\ b^3 - 1,\ c^5 - 1,\ {(ab)}^2 - 1,\ {(ac)}^2 - 1,\ {(bc)}^2 - 1\}
  \end{align*}
\end{frame}

\begin{frame}
  \frametitle{Comparación}
  Tiempo en calcular una base de Gröbner
  \noindent \begin{tikzpicture}
    \begin{axis}[
        ybar = 0.5pt, % Tipo de gráfico: barras verticales
        width = \textwidth,
        height = 0.8\textheight,
        bar width = 0.01\textwidth, % Ancho de cada barra
        symbolic x coords = {FK2, FK3, FK4, tri1, tri2, tri3, tri4, tri5, tri6, tri7, tri8, tri9, tri10, tri11, tri12, tri13, trit3, trit4, trit5}, % Etiquetas en el eje x
        xtick = data, % Mostrar marcas en el eje x para cada dato
        ylabel = {Tiempo en segundos}, % Etiqueta del eje y
        %xlabel = {Categorías}, % Etiqueta del eje x
        xticklabel style = {rotate = 45}, % Rotar etiquetas
        ymin = 0, % Mínimo del eje y
        ymax = 215,
        enlarge x limits = 0.03, % Espaciado en los bordes
        legend style = {at = {(0.024, 0.94)}, anchor = north west}, % Estilo de la leyenda
        legend entries = {\texttt{Buchberger}, \texttt{F4}, \texttt{operator\_gb}}, % Entradas de la leyenda
        cycle list = {{brown,fill = brown!30}, {red,fill = red!30}, {blue,fill = blue!30}}, % Colores
    ]
      % Datos para cada conjunto
      \addplot coordinates {(FK2, 0.00) (FK3, 0.00) (FK4, 0.01) (tri1, 0.26) (tri10, 0.01) (tri11, 0.03) (tri12, 3.27) (tri13, 0.01) (tri2, 6.01) (tri3, 0.02) (tri4, 0.04) (tri5, 0.01) (tri6, 6.53) (tri7, 33.33) (tri8, 0.09) (tri9, 0.00) (trit3, 0.01) (trit4, 0.14) (trit5, 77.16)};
      \addplot coordinates {(FK2, 0.01) (FK3, 0.01) (FK4, 0.08) (tri1, 0.08) (tri10, 0.01) (tri11, 0.02) (tri12, 0.21) (tri13, 0.01) (tri2, 0.29) (tri3, 0.04) (tri4, 0.07) (tri5, 0.02) (tri6, 1.78) (tri7, 1.81) (tri8, 0.06) (tri9, 0.01) (trit3, 0.05) (trit4, 0.49) (trit5, 207.84)};
      \addplot coordinates {(FK2, 1.70) (FK3, 0.80) (FK4, 0.87) (tri1, 0.98) (tri10, 0.85) (tri11, 0.88) (tri12, 1.04) (tri13, 0.83) (tri2, 1.16) (tri3, 0.90) (tri4, 0.94) (tri5, 0.87) (tri6, 1.27) (tri7, 1.45) (tri8, 0.91) (tri9, 0.82) (trit3, 0.87) (trit4, 1.04) (trit5, 5.21)};
    \end{axis}
  \end{tikzpicture}
\end{frame}


\begin{frame}
  \frametitle{Comparación}
  Tiempo en calcular una base de Gröbner viendo la parte de más abajo
  \noindent \begin{tikzpicture}
    \begin{axis}[
        ybar = 0.5pt, % Tipo de gráfico: barras verticales
        width = \textwidth,
        height = 0.8\textheight,
        bar width = 0.01\textwidth, % Ancho de cada barra
        symbolic x coords = {FK2, FK3, FK4, tri1, tri2, tri3, tri4, tri5, tri6, tri7, tri8, tri9, tri10, tri11, tri12, tri13, trit3, trit4, trit5}, % Etiquetas en el eje x
        xtick = data, % Mostrar marcas en el eje x para cada dato
        ylabel = {Tiempo en segundos}, % Etiqueta del eje y
        %xlabel = {Categorías}, % Etiqueta del eje x
        xticklabel style = {rotate = 45}, % Rotar etiquetas
        ymin = 0, % Mínimo del eje y
        ymax = 10,
        enlarge x limits = 0.03, % Espaciado en los bordes
        legend style = {at = {(0.024, 0.94)}, anchor = north west}, % Estilo de la leyenda
        legend entries = {\texttt{Buchberger}, \texttt{F4}, \texttt{operator\_gb}}, % Entradas de la leyenda
        cycle list = {{brown,fill = brown!30}, {red,fill=red!30}, {blue,fill=blue!30}}, % Colores
    ]
    ]
      % Datos para cada conjunto
      \addplot coordinates {(FK2, 0.00) (FK3, 0.00) (FK4, 0.01) (tri1, 0.26) (tri10, 0.01) (tri11, 0.03) (tri12, 3.27) (tri13, 0.01) (tri2, 6.01) (tri3, 0.02) (tri4, 0.04) (tri5, 0.01) (tri6, 6.53) (tri7, 33.33) (tri8, 0.09) (tri9, 0.00) (trit3, 0.01) (trit4, 0.14) (trit5, 77.16)};
      \addplot coordinates {(FK2, 0.01) (FK3, 0.01) (FK4, 0.08) (tri1, 0.08) (tri10, 0.01) (tri11, 0.02) (tri12, 0.21) (tri13, 0.01) (tri2, 0.29) (tri3, 0.04) (tri4, 0.07) (tri5, 0.02) (tri6, 1.78) (tri7, 1.81) (tri8, 0.06) (tri9, 0.01) (trit3, 0.05) (trit4, 0.49) (trit5, 207.84)};
      \addplot coordinates {(FK2, 1.70) (FK3, 0.80) (FK4, 0.87) (tri1, 0.98) (tri10, 0.85) (tri11, 0.88) (tri12, 1.04) (tri13, 0.83) (tri2, 1.16) (tri3, 0.90) (tri4, 0.94) (tri5, 0.87) (tri6, 1.27) (tri7, 1.45) (tri8, 0.91) (tri9, 0.82) (trit3, 0.87) (trit4, 1.04) (trit5, 5.21)};
    \end{axis}
  \end{tikzpicture}
\end{frame}

\begin{frame} % Ver de hacer un gráfico que tenga en el eje x la cantidad de cores y tenga una linea por cada caso
  \frametitle{Comparación}
  Paralelismo
  \noindent \begin{tikzpicture}
    \begin{axis}[
        ybar = 0.5pt, % Tipo de gráfico: barras verticales
        width = \textwidth,
        height = 0.8\textheight,
        bar width = 0.003\textwidth,
        symbolic x coords = {FK2, FK3, FK4, tri1, tri2, tri3, tri4, tri5, tri6, tri7, tri8, tri9, tri10, tri11, tri12, tri13, trit3, trit4, trit5}, % Etiquetas en el eje x
        xtick = data, % Mostrar marcas en el eje x para cada dato
        ylabel = {Tiempo en segundos}, % Etiqueta del eje y
        %xlabel = {Cantidad de hilos}, % Etiqueta del eje x
        xticklabel style = {rotate = 45}, % Rotar etiquetas
        ymin = 0, % Mínimo del eje y
        ymax = 215, % Máximo del eje y
        enlarge x limits = 0.03, % Espaciado en los bordes
        legend style = {at = {(0.024, 0.94)}, anchor = north west}, % Estilo de la leyenda
        legend entries = {1, 2, 4, 6, 8, 16}, % Entradas de la leyenda
    ]
      % Datos para cada conjunto
      \addplot coordinates {(FK2, 0.00) (FK3, 0.00) (FK4, 0.07) (tri1, 0.10) (tri10, 0.01) (tri11, 0.02) (tri12, 0.32) (tri13, 0.01) (tri2, 0.53) (tri3, 0.03) (tri4, 0.07) (tri5, 0.01) (tri6, 2.20) (tri7, 2.19) (tri8, 0.07) (tri9, 0.00) (trit3, 0.04) (trit4, 0.47) (trit5, 213.58)};
      \addplot coordinates {(FK2, 0.00) (FK3, 0.00) (FK4, 0.06) (tri1, 0.08) (tri10, 0.01) (tri11, 0.02) (tri12, 0.28) (tri13, 0.00) (tri2, 0.42) (tri3, 0.03) (tri4, 0.06) (tri5, 0.01) (tri6, 1.98) (tri7, 2.02) (tri8, 0.06) (tri9, 0.00) (trit3, 0.04) (trit4, 0.47) (trit5, 212.76)};
      \addplot coordinates {(FK2, 0.00) (FK3, 0.00) (FK4, 0.06) (tri1, 0.07) (tri10, 0.01) (tri11, 0.01) (tri12, 0.22) (tri13, 0.00) (tri2, 0.36) (tri3, 0.03) (tri4, 0.06) (tri5, 0.01) (tri6, 1.82) (tri7, 1.92) (tri8, 0.06) (tri9, 0.00) (trit3, 0.04) (trit4, 0.45) (trit5, 209.93)};
      \addplot coordinates {(FK2, 0.00) (FK3, 0.00) (FK4, 0.06) (tri1, 0.06) (tri10, 0.01) (tri11, 0.01) (tri12, 0.22) (tri13, 0.01) (tri2, 0.32) (tri3, 0.03) (tri4, 0.06) (tri5, 0.01) (tri6, 1.86) (tri7, 1.84) (tri8, 0.05) (tri9, 0.00) (trit3, 0.04) (trit4, 0.45) (trit5, 212.14)};
      \addplot coordinates {(FK2, 0.00) (FK3, 0.00) (FK4, 0.07) (tri1, 0.06) (tri10, 0.01) (tri11, 0.01) (tri12, 0.20) (tri13, 0.00) (tri2, 0.31) (tri3, 0.03) (tri4, 0.06) (tri5, 0.01) (tri6, 1.83) (tri7, 1.81) (tri8, 0.05) (tri9, 0.00) (trit3, 0.04) (trit4, 0.45) (trit5, 208.28)};
      \addplot coordinates {(FK2, 0.00) (FK3, 0.00) (FK4, 0.07) (tri1, 0.06) (tri10, 0.01) (tri11, 0.01) (tri12, 0.20) (tri13, 0.00) (tri2, 0.29) (tri3, 0.03) (tri4, 0.05) (tri5, 0.01) (tri6, 1.79) (tri7, 1.80) (tri8, 0.05) (tri9, 0.00) (trit3, 0.04) (trit4, 0.46) (trit5, 208.62)};
    \end{axis}
  \end{tikzpicture}
\end{frame}

\begin{frame}
  \frametitle{Comparación}
  Paralelismo sin trit5
  \noindent \begin{tikzpicture}
    \begin{axis}[
        ybar = 0.5pt, % Tipo de gráfico: barras verticales
        width = \textwidth,
        height = 0.8\textheight,
        bar width = 0.003\textwidth,
        symbolic x coords = {FK2, FK3, FK4, tri1, tri2, tri3, tri4, tri5, tri6, tri7, tri8, tri9, tri10, tri11, tri12, tri13, trit3, trit4}, % Etiquetas en el eje x
        xtick = data, % Mostrar marcas en el eje x para cada dato
        ylabel = {Tiempo en segundos}, % Etiqueta del eje y
        %xlabel = {Cantidad de hilos}, % Etiqueta del eje x
        xticklabel style = {rotate = 45}, % Rotar etiquetas
        ymin = 0, % Mínimo del eje y
        ymax = 2.5, % Máximo del eje y
        enlarge x limits = 0.03, % Espaciado en los bordes
        legend style = {at = {(0.024, 0.94)}, anchor = north west}, % Estilo de la leyenda
        legend entries = {1, 2, 4, 6, 8, 16}, % Entradas de la leyenda
    ]
      % Datos para cada conjunto
      \addplot coordinates {(FK2, 0.00) (FK3, 0.00) (FK4, 0.07) (tri1, 0.10) (tri10, 0.01) (tri11, 0.02) (tri12, 0.32) (tri13, 0.01) (tri2, 0.53) (tri3, 0.03) (tri4, 0.07) (tri5, 0.01) (tri6, 2.20) (tri7, 2.19) (tri8, 0.07) (tri9, 0.00) (trit3, 0.04) (trit4, 0.47)};
      \addplot coordinates {(FK2, 0.00) (FK3, 0.00) (FK4, 0.06) (tri1, 0.08) (tri10, 0.01) (tri11, 0.02) (tri12, 0.28) (tri13, 0.00) (tri2, 0.42) (tri3, 0.03) (tri4, 0.06) (tri5, 0.01) (tri6, 1.98) (tri7, 2.02) (tri8, 0.06) (tri9, 0.00) (trit3, 0.04) (trit4, 0.47)};
      \addplot coordinates {(FK2, 0.00) (FK3, 0.00) (FK4, 0.06) (tri1, 0.07) (tri10, 0.01) (tri11, 0.01) (tri12, 0.22) (tri13, 0.00) (tri2, 0.36) (tri3, 0.03) (tri4, 0.06) (tri5, 0.01) (tri6, 1.82) (tri7, 1.92) (tri8, 0.06) (tri9, 0.00) (trit3, 0.04) (trit4, 0.45)};
      \addplot coordinates {(FK2, 0.00) (FK3, 0.00) (FK4, 0.06) (tri1, 0.06) (tri10, 0.01) (tri11, 0.01) (tri12, 0.22) (tri13, 0.01) (tri2, 0.32) (tri3, 0.03) (tri4, 0.06) (tri5, 0.01) (tri6, 1.86) (tri7, 1.84) (tri8, 0.05) (tri9, 0.00) (trit3, 0.04) (trit4, 0.45)};
      \addplot coordinates {(FK2, 0.00) (FK3, 0.00) (FK4, 0.07) (tri1, 0.06) (tri10, 0.01) (tri11, 0.01) (tri12, 0.20) (tri13, 0.00) (tri2, 0.31) (tri3, 0.03) (tri4, 0.06) (tri5, 0.01) (tri6, 1.83) (tri7, 1.81) (tri8, 0.05) (tri9, 0.00) (trit3, 0.04) (trit4, 0.45)};
      \addplot coordinates {(FK2, 0.00) (FK3, 0.00) (FK4, 0.07) (tri1, 0.06) (tri10, 0.01) (tri11, 0.01) (tri12, 0.20) (tri13, 0.00) (tri2, 0.29) (tri3, 0.03) (tri4, 0.05) (tri5, 0.01) (tri6, 1.79) (tri7, 1.80) (tri8, 0.05) (tri9, 0.00) (trit3, 0.04) (trit4, 0.46)};
    \end{axis}
  \end{tikzpicture}
\end{frame}

\end{document}
