\documentclass{beamer}

\usepackage{fontspec}

\usefonttheme{serif}

\usepackage{polyglossia}
\setdefaultlanguage{spanish}

\usepackage{relsize}
\usepackage{xspace}

%\usepackage{tikz}
%\usepackage{pgf} % Para dibujar la cuadrícula
%% Definir un fondo con cuadrícula
%\usebackgroundtemplate{
%  \begin{tikzpicture}
%    \draw[step=0.5cm,gray,very thin] (0,0) grid (\paperwidth,\paperheight);
%  \end{tikzpicture}
%}

\newcommand\cpp{C\nolinebreak[4]\hspace{-.05em}\raisebox{.4ex}{\relsize{-3}{\textbf{++}}}\xspace}


\title{Implementación de bases de Gröbner no conmutativas en C++ con un poquito de paralelismo}
\author{Iván Renison}
\institute{Facultad de Matemática, Astronomía, Física y Computación\\
  Universidad Nacional de Córdoba}
\date{2025-03-06}

% \pgfdeclareimage[height=0.5cm]{university-logo}{logos/unc.jpg}
% \logo{\pgfuseimage{university-logo}}


\begin{document}

\begin{frame}
  \title{Implementación de bases de Gröbner no conmutativas en \cpp con un poquito de paralelismo}
  \titlepage
\end{frame}

\begin{frame}
  \frametitle{Polinomios}
\end{frame}

\begin{frame}
  \frametitle{Polinomios no conmutativos}
\end{frame}

\begin{frame}
  \frametitle{Operaciones entre polinomios no conmutativos}
\end{frame}

\begin{frame}
  \frametitle{Problema de pertenencia al ideal}
\end{frame}

\begin{frame}
  \frametitle{Ejemplo de problema de pertenencia al ideal}
\end{frame}

\begin{frame}
  \frametitle{Ideales}
\end{frame}

\begin{frame}
  \frametitle{Bases de Gröbner}
\end{frame}

\begin{frame}
  \frametitle{Algoritmos}
\end{frame}

\begin{frame}
  \frametitle{Implementaciones previas}
\end{frame}

\begin{frame}
  \frametitle{Mi librería}
\end{frame}

\begin{frame}
  \frametitle{Comparación}
\end{frame}

\begin{frame}
  \frametitle{Trabajos futuros}
\end{frame}

\end{document}
