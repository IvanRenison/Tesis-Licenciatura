\documentclass[spanish, aspectratio=169]{beamer}
%\usetheme{moloch}

\usepackage{fontspec}

\usefonttheme{serif}

\usepackage{polyglossia}
\setdefaultlanguage{spanish}

\usepackage[math-style=ISO,bold-style=ISO]{unicode-math}


\usepackage{relsize}
\usepackage{xspace}

%\usepackage{tikz}
%\usepackage{pgf} % Para dibujar la cuadrícula
%% Definir un fondo con cuadrícula
%\usebackgroundtemplate{
%  \begin{tikzpicture}
%    \draw[step=0.5cm,gray,very thin] (0,0) grid (\paperwidth,\paperheight);
%  \end{tikzpicture}
%}


\newcommand\cpp{C\nolinebreak[4]\hspace{-.05em}\raisebox{.4ex}{\relsize{-3}{\textbf{++}}}\xspace}


\title{Implementación de bases de Gröbner no conmutativas en C++ con un poquito de paralelismo}
\author{Iván Renison}
\institute{Facultad de Matemática, Astronomía, Física y Computación\\
  Universidad Nacional de Córdoba}
\date{2025-03-06}

% \pgfdeclareimage[height=0.5cm]{university-logo}{logos/unc.jpg}
% \logo{\pgfuseimage{university-logo}}


\begin{document}

\begin{frame}
  \title{Implementación de bases de Gröbner no conmutativas en \cpp con un poquito de paralelismo}
  \titlepage
\end{frame}

\begin{frame}
  \frametitle{Polinomios}

  Algunos polinomios:
  \begin{itemize}
    \item $5 + 3 y - 2 x y + x^3 y^5$
    \item $y^2 + x - x^2y^2$
  \end{itemize}

  Fijemos:
  \begin{itemize}
    \item Un alfabeto $X$
    \item Un cuerpo $K$
  \end{itemize}

  El conjunto de polinomios se denota $K[X]$.

  El producto es conmutativo. Por ejemplo:
  \begin{itemize}
    \item $xy = yx$
    \item $y^2 + x - x^2y^2 = yy + x - yxxy$
  \end{itemize}

\end{frame}

\begin{frame}
  \frametitle{Polinomios no conmutativos}
  \begin{align*}
    f_0 &= a \\
    f_1 &= ab + bc \\
    f_2 &= 3 abb - 2 acab + 4 bcca \\
    f_3 &= ba + ca \\
    f_4 &= aba - bba \\
    f_5 &= abc + cba \\
    f_6 &= -aba + cba - 2 baba + 2 ccba
  \end{align*}

  Se denotan:
  \begin{itemize}
    \item $K⟨X⟩$ el conjunto de polinomios no conmutativos
    \item $⟨X⟩$ el conjunto de monomios
  \end{itemize}
\end{frame}

\begin{frame}
  \frametitle{Ordenes monomiales}
  $≤_{deglex}$ ordena a $⟨X⟩$ primero por largo y después por orden lexicográfico.
\end{frame}

\begin{frame}
  \frametitle{Problema de pertenencia al ideal}
  Dados $G ⊆ K⟨X⟩$ y $f ∈ K⟨X⟩$, determinar si vale que
  \[ ∃g_1, …, g_n ∈ G, f_1, …, f_n, f'_1, …, f'_n ∈ K⟨X⟩ : f = ∑_{i = 1}^n f_i g_i f'_i \text{.}\]
\end{frame}

\begin{frame}
  \frametitle{Ejemplo de problema de pertenencia al ideal}
  \begin{align*}
    g_0 &= f_1 = ab + bc \\
    g_1 &= f_3 = ba + ca \\
    G &= \{g_0, g_1\}
  \end{align*}

  Para $f_4 = aba - bba$ si vale:
  \begin{align*}
    g_0 a - b g_1 &= (ab + bc)a - b(ba + ca) \\
      &= aba + bca - bba - bca \\
      &= aba - bba \\
      &= f_4
  \end{align*}

  Para $f_5 = abc + cba$ no vale.

\end{frame}

\begin{frame}
  \frametitle{Ideales}
  Sea $G ⊆ K⟨X⟩$, se define
  \[ (G) = \{∑_{i = 1}^n c_i g_i c_i' : g_1, …, g_n ∈ G, c_1, …, c_n, c_1', …, c_n' ∈ K⟨X⟩\} \]
  A $(G)$ se lo llama el ideal generado por $G$.
\end{frame}

\begin{frame}
  \frametitle{Teorema}
  Sean $G ⊆ K⟨X⟩$, $f ∈ K⟨X⟩$, $g ∈ G$, $a, b ∈ ⟨X⟩$ y $c ∈ ⟨X⟩$. Entonces
  \[ f ∈ (G) ⇔ f + c a g b ∈ (G) \]
\end{frame}


\begin{frame}
  \frametitle{Reducciones}
  \begin{align*}
    g_0 &= f_1 = ab + bc \\
    g_1 &= f_3 = ba + ca \\
    G &= \{g_0, g_1\} \\
    & \\
    f_4 &= aba - bba \\
    f_5 &= abc + cba \\
    f_6 &= -aba + cba - 2 baba + 2 ccba
  \end{align*}
\end{frame}

\begin{frame}
  \frametitle{Algoritmo de Buchberger}

  La idea de Buchberger es ir agregando los S-polinomios reducidos
\end{frame}

\begin{frame}
  \frametitle{Bases de Gröbner}
  $G$ es una base de Gröbner $⇔ ∀f ∈ K⟨X⟩ : f ∈ (G) ⇔ f$ se puede reducir
  Si agregamos S-polinomios hasta que no queda ninguno entonces obtenemos una base de Gröbner
\end{frame}

\begin{frame}
  \frametitle{Implementaciones previas}
\end{frame}

\begin{frame}
  \frametitle{Mi librería}
\end{frame}

\begin{frame}
  \frametitle{Comparación}
\end{frame}

\begin{frame}
  \frametitle{Repaso}
\end{frame}

\begin{frame}
  \frametitle{Trabajos futuros}
\end{frame}

\end{document}
