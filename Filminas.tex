\documentclass[spanish, aspectratio=169, hidecontrols]{beamer}
%\usetheme{moloch}
\setbeamertemplate{navigation symbols}{}

\usepackage{fontspec}

\usefonttheme{serif}

\usepackage{polyglossia}
\setdefaultlanguage{spanish}

\usepackage[math-style=ISO,bold-style=ISO]{unicode-math}

\usepackage{minted}
\setminted{autogobble}

\usepackage{relsize}
\usepackage{xspace}

\usepackage{pgfplots} % Para los gráficos

%\usepackage{tikz}
%\usepackage{pgf} % Para dibujar la cuadrícula
%% Definir un fondo con cuadrícula
%\usebackgroundtemplate{
%  \begin{tikzpicture}
%    \draw[step=0.5cm,gray,very thin] (0,0) grid (\paperwidth,\paperheight);
%  \end{tikzpicture}
%}


\newcommand\cpp{C\nolinebreak[4]\hspace{-.05em}\raisebox{.4ex}{\relsize{-3}{\textbf{++}}}\xspace}


\title{Implementación de bases de Gröbner no conmutativas en C++ con un poquito de paralelismo}
\author{Iván Renison}
\institute{Facultad de Matemática, Astronomía, Física y Computación\\
  Universidad Nacional de Córdoba}
\date{2025-03-06}

% \pgfdeclareimage[height=0.5cm]{university-logo}{logos/unc.jpg}
% \logo{\pgfuseimage{university-logo}}


\begin{document}

% Cosas que tengo que tener listas:
% El repositorio de la librería abierto en el navegador y en vscode
% La terminal abierta en el directorio de la librería

\begin{frame}
  \title{Implementación de bases de Gröbner no conmutativas en \cpp con un poquito de paralelismo}
  \titlepage
\end{frame}

\begin{frame}
  \frametitle{Polinomios}

  \pause

  Algunos polinomios:
  \begin{itemize}
    \item $5 + 3 y - 2 x y + x^3 y^5$
    \item $y^2 + x - x^2y^2$
  \end{itemize}

  Fijemos:
  \begin{itemize}
    \item Un alfabeto $X$
    \item Un cuerpo $K$
  \end{itemize}

  El conjunto de polinomios se denota $K[X]$.

  El producto es conmutativo. Por ejemplo:
  \begin{itemize}
    \item $xy = yx$
    \item $y^2 + x - x^2y^2 = yy + x - yxxy$
  \end{itemize}

\end{frame}

\begin{frame}
  \frametitle{Polinomios no conmutativos}
  \begin{flalign*}
    & f_0 = a &&\\
    & f_1 = ab + bc &&\\
    & f_2 = 3 abb - 2 acab + \frac{4}{3} bcca &&\\
    & f_3 = ba + ca &&\\
    & f_4 = aba - bba &&\\
    & f_5 = abc + cba &&\\
    & f_6 = -aba + cba - 2 baba + 2 ccba &&
  \end{flalign*}

  \pause

  Se denotan:
  \begin{itemize}
    \item \alert{$K⟨X⟩$} el conjunto de polinomios no conmutativos
    \item $⟨X⟩$ el conjunto de monomios
  \end{itemize}
\end{frame}

\begin{frame}
  \frametitle{Ordenes monomiales}
  $≤_{deglex}$ ordena a $⟨X⟩$ primero por largo y después por orden lexicográfico.
  \pause
  \begin{flalign*}
    & f_0 = a &&\\
    & f_1 = ab + bc &&\\
    & f_2 = 3 abb - 2 acab + \frac{4}{3} bcca &&\\
    & f_3 = ba + ca &&\\
    & f_4 = aba - bba &&\\
    & f_5 = abc + cba &&\\
    & f_6 = -aba + cba - 2 baba + 2 ccba &&
  \end{flalign*}
\end{frame}

\begin{frame}
  \frametitle{Problema de pertenencia al ideal}
  Dados $G ⊆ K⟨X⟩$ y $f ∈ K⟨X⟩$, determinar si vale que
  \[ ∃g_1, …, g_n ∈ G, f_1, …, f_n, f'_1, …, f'_n ∈ K⟨X⟩ : \alert{f = ∑_{i = 1}^n f_i g_i f'_i} \text{.}\]
  \pause
  Ejemplo
  \begin{align*}
    g_0 &= f_1 = ab + bc \\
    g_1 &= f_3 = ba + ca \\
    G &= \{g_0, g_1\}
  \end{align*}

  Para $f_4 = aba - bba$ si vale\pause, porque es igual a $g_0 a - b g_1$:
  \pause
  \begin{align*}
    g_0 a - b g_1 &= (ab + bc)a - b(ba + ca) \\
      &= aba + bca - bba - bca \\
      &= aba - bba \\
      &= f_4
  \end{align*}

  Para $f_5 = abc + cba$ no vale.

\end{frame}

\begin{frame}
  \frametitle{Ideales}
  \pause
  Sea $G ⊆ K⟨X⟩$, se define
  \[ (G) = \{∑_{i = 1}^n c_i g_i c_i' : g_1, …, g_n ∈ G, c_1, …, c_n, c_1', …, c_n' ∈ K⟨X⟩\} \]
  A $(G)$ se lo llama el ideal generado por $G$.
\end{frame}

\begin{frame}
  \frametitle{Teorema}
  \pause
  Sean $G ⊆ K⟨X⟩$, $f ∈ K⟨X⟩$, $g ∈ G$, $a, b ∈ ⟨X⟩$ y $c ∈ ⟨X⟩$. Entonces
  \[ f ∈ (G) ⇔ f + c a g b ∈ (G) \]
\end{frame}


\begin{frame}
  \frametitle{Reducciones}
  \begin{align*}
    g_0 &= f_1 = ab + bc \\
    g_1 &= f_3 = ba + ca \\
    G &= \{g_0, g_1\} \\
    & \\
    f_4 &= aba - bba \\
    f_5 &= abc + cba \\
    f_6 &= -aba + cba - 2 baba + 2 ccba
  \end{align*}
\end{frame}

\begin{frame}
  \frametitle{Algoritmo de Buchberger}
  \pause

  La idea de Buchberger es ir agregando los S-polinomios reducidos
\end{frame}

\begin{frame}
  \frametitle{Bases de Gröbner}
  $G$ es una base de Gröbner $⇔ ∀f ∈ K⟨X⟩ : f ∈ (G) ⇔ f$ se puede reducir \\

  \pause

  Si agregamos S-polinomios hasta que no queda ninguno entonces \\ obtenemos una base de Gröbner
\end{frame}

\begin{frame}
  \frametitle{Algoritmo F4}
  \pause

  F4 hace lo mismo que Buchberger pero de a muchos polinomios al mismo tiempo usando álgebra lineal
\end{frame}

\begin{frame}
  \frametitle{Implementaciones previas}
  \pause
  De Buchberger:
  \begin{itemize}
    \item Paquete GBNP de GAP
    \item Implementación dentro de Singular
    \item Implementación dentro de NCAlgebra
    \item Paquete \texttt{OperatorGB} de Mathematica
  \end{itemize}
  \pause
  De F4:
  \begin{itemize}
    \item Implementación dentro de Magma
    \item Paquete \texttt{operator\_gb} de Python y SageMath
  \end{itemize}
\end{frame}

\begin{frame}
  \frametitle{Mi librería}
  Librería \texttt{ncgb} de \cpp
  \pause
  \begin{itemize}
    \item Implementa estructuras para monomios y polinomios con sus operaciones básicas
    \pause
    \item Implementa Buchberger y F4
    \pause
    \item Funciona para un \alert{cuerpo arbitrario}
    \pause
    \item Para los racionales usa la libraría FLINT para las matrices
    \pause
    \item Incluye \alert{representación de cofactores} (``\textit{reconstrucción de la respuesta}'')
    \pause
    \item Tiene un poquito de paralelismo
    \pause
    \item Falta implementar la optimización de Faugère-Lachartre en F4
  \end{itemize}
\end{frame}

\begin{frame}
  \frametitle{Ejemplo de mi librería}
  \begin{align*}
    g_0 &= f_1 = ab + bc \\
    g_1 &= f_3 = ba + ca \\
    G &= \{g_0, g_1\}
  \end{align*}

  Dijimos que:
  \begin{itemize}
    \item Para $f_4 = aba - bba$ si vale.
    \item Para $f_5 = abc + cba$ no vale.
  \end{itemize}
\end{frame}

\begin{frame}[fragile]
  \frametitle{Comparación}
  \begin{columns}
    \begin{column}{0.5\textwidth}
      Para los racionales:
      \begin{minted}{text}
        Average times:
        Buch: 0.0041429424s
        F4: 0.0105502963s
        GB: 0.8995876765s
        Max times:
        Buch: 0.1320753098s
        F4: 0.0925185680s
        GB: 4.6477575302s
      \end{minted}
    \end{column}
    \begin{column}{0.5\textwidth}
      Para la aritmética modular:
      \begin{minted}{text}
        Average times:
        Buch_Zp: 0.0028817248s
        F4_Zp: 0.0072190428s
        Max times:
        Buch_Zp: 0.0779249668s
        F4_Zp: 0.0799708366s
      \end{minted}
    \end{column}
  \end{columns}
\end{frame}

\begin{frame}
  \frametitle{Comparación}
    \begin{tiny}
    \setlength{\jot}{0pt}
    \begin{align*}
      \text{FK2} =& \{a^2\} \\
      \text{FK3} =& \{a^2,\ b^2,\ c^2,\ ac + ba + cb,\ ab + bc + ca\} \\
      \text{FK4} =& \{a^2,\ b^2,\ c^2,\ d^2,\ e^2,\ f^2,\ ac + ba + cb,\ ae + da + ed,\ bf + db + fd,\ cf + ec + fe,\ ab + bc + ca,\ ad + de + ea, \\
        & bd + df + fb,\ ce + ef + fc,\ cd + dc,\ be + eb,\ af + fa\} \\
      \text{tri1} =& \{a^2 - 1,\ b^3 - 1,\ {(ababab^2ab^2)}^2 - 1\} \\
      \text{tri2} =& \{a^2 - 1,\ b^3 - 1,\ {(ababab^2)}^3 - 1\} \\
      \text{tri3} =& \{a^3 - 1,\ b^3 - 1,\ {(abab^2)}^2 - 1\} \\
      \text{tri4} =& \{a^3 - 1,\ b^3 - 1,\ {(abaab^2)}^2 - 1\} \\
      \text{tri5} =& \{a^2 - 1,\ b^5 - 1,\ {(abab^2)}^2 - 1\} \\
      \text{tri6} =& \{a^2 - 1,\ b^5 - 1,\ {(ababab^4)}^2 - 1\} \\
      \text{tri7} =& \{a^2 - 1,\ b^5 - 1,\ {(abab^2ab^4)}^2 - 1\} \\
      \text{tri8} =& \{a^2 - 1,\ b^2 - 1,\ {(ababab^3)}^2 - 1\} \\
      \text{tri9} =& \{a^2 - 1,\ b^3 - 1,\ {(abab^2)}^2 - 1\} \\
      \text{tri10} =& \{a^2 - 1,\ b^3 - 1,\ {(ababab^2)}^2 - 1\} \\
      \text{tri11} =& \{a^2 - 1,\ b^3 - 1,\ {(abababab^2)}^2 - 1\} \\
      \text{tri12} =& \{a^2 - 1,\ b^3 - 1,\ {(ababab^2abab^2)}^2 - 1\} \\
      \text{tri13} =& \{a^2 - 1,\ b^3 - 1,\ {(babababab^2ab^2)}^2 - 1\} \\
      \text{trit3} =& \{a^3 - 1,\ b^3 - 1,\ c^3 - 1,\ {(ab)}^2 - 1,\ {(ac)}^2 - 1,\ {(bc)}^2 - 1\} \\
      \text{trit4} =& \{a^3 - 1,\ b^3 - 1,\ c^4 - 1,\ {(ab)}^2 - 1,\ {(ac)}^2 - 1,\ {(bc)}^2 - 1\} \\
      \text{trit5} =& \{a^3 - 1,\ b^3 - 1,\ c^5 - 1,\ {(ab)}^2 - 1,\ {(ac)}^2 - 1,\ {(bc)}^2 - 1\}
    \end{align*}
    \end{tiny}
\end{frame}

\begin{frame}
  \frametitle{Comparación}
  Tiempo en calcular una base
  \noindent \begin{tikzpicture}
    \begin{axis}[
        ybar = 0.5pt, % Tipo de gráfico: barras verticales
        width = \textwidth,
        height = 0.8\textheight,
        bar width = 0.01\textwidth, % Ancho de cada barra
        symbolic x coords = {FK2, FK3, FK4, tri1, tri2, tri3, tri4, tri5, tri6, tri7, tri8, tri9, tri10, tri11, tri12, tri13, trit3, trit4, trit5}, % Etiquetas en el eje x
        xtick = data, % Mostrar marcas en el eje x para cada dato
        ylabel = {Tiempo en segundos}, % Etiqueta del eje y
        %xlabel = {Categorías}, % Etiqueta del eje x
        xticklabel style = {rotate = 45}, % Rotar etiquetas
        ymin = 0, % Mínimo del eje y
        ymax = 215,
        enlarge x limits = 0.03, % Espaciado en los bordes
        legend style = {at = {(0.024, 0.94)}, anchor = north west}, % Estilo de la leyenda
        legend entries = {\texttt{Buchberger}, \texttt{F4}, \texttt{operator\_gb}}, % Entradas de la leyenda
        cycle list = {{brown,fill = brown!30}, {red,fill = red!30}, {blue,fill = blue!30}}, % Colores
    ]
      % Datos para cada conjunto
      \addplot coordinates {(FK2, 0.00) (FK3, 0.00) (FK4, 0.01) (tri1, 0.26) (tri10, 0.01) (tri11, 0.03) (tri12, 3.27) (tri13, 0.01) (tri2, 6.01) (tri3, 0.02) (tri4, 0.04) (tri5, 0.01) (tri6, 6.53) (tri7, 33.33) (tri8, 0.09) (tri9, 0.00) (trit3, 0.01) (trit4, 0.14) (trit5, 77.16)};
      \addplot coordinates {(FK2, 0.01) (FK3, 0.01) (FK4, 0.08) (tri1, 0.08) (tri10, 0.01) (tri11, 0.02) (tri12, 0.21) (tri13, 0.01) (tri2, 0.29) (tri3, 0.04) (tri4, 0.07) (tri5, 0.02) (tri6, 1.78) (tri7, 1.81) (tri8, 0.06) (tri9, 0.01) (trit3, 0.05) (trit4, 0.49) (trit5, 207.84)};
      \addplot coordinates {(FK2, 1.70) (FK3, 0.80) (FK4, 0.87) (tri1, 0.98) (tri10, 0.85) (tri11, 0.88) (tri12, 1.04) (tri13, 0.83) (tri2, 1.16) (tri3, 0.90) (tri4, 0.94) (tri5, 0.87) (tri6, 1.27) (tri7, 1.45) (tri8, 0.91) (tri9, 0.82) (trit3, 0.87) (trit4, 1.04) (trit5, 5.21)};
    \end{axis}
  \end{tikzpicture}
\end{frame}

\begin{frame}
  \frametitle{Comparación}
  Tiempo en calcular una base viendo la parte de más abajo
  \noindent \begin{tikzpicture}
    \begin{axis}[
        ybar = 0.5pt, % Tipo de gráfico: barras verticales
        width = \textwidth,
        height = 0.8\textheight,
        bar width = 0.01\textwidth, % Ancho de cada barra
        symbolic x coords = {FK2, FK3, FK4, tri1, tri2, tri3, tri4, tri5, tri6, tri7, tri8, tri9, tri10, tri11, tri12, tri13, trit3, trit4, trit5}, % Etiquetas en el eje x
        xtick = data, % Mostrar marcas en el eje x para cada dato
        ylabel = {Tiempo en segundos}, % Etiqueta del eje y
        %xlabel = {Categorías}, % Etiqueta del eje x
        xticklabel style = {rotate = 45}, % Rotar etiquetas
        ymin = 0, % Mínimo del eje y
        ymax = 10,
        enlarge x limits = 0.03, % Espaciado en los bordes
        legend style = {at = {(0.024, 0.94)}, anchor = north west}, % Estilo de la leyenda
        legend entries = {\texttt{Buchberger}, \texttt{F4}, \texttt{operator\_gb}}, % Entradas de la leyenda
        cycle list = {{brown,fill = brown!30}, {red,fill=red!30}, {blue,fill=blue!30}}, % Colores
    ]
    ]
      % Datos para cada conjunto
      \addplot coordinates {(FK2, 0.00) (FK3, 0.00) (FK4, 0.01) (tri1, 0.26) (tri10, 0.01) (tri11, 0.03) (tri12, 3.27) (tri13, 0.01) (tri2, 6.01) (tri3, 0.02) (tri4, 0.04) (tri5, 0.01) (tri6, 6.53) (tri7, 33.33) (tri8, 0.09) (tri9, 0.00) (trit3, 0.01) (trit4, 0.14) (trit5, 77.16)};
      \addplot coordinates {(FK2, 0.01) (FK3, 0.01) (FK4, 0.08) (tri1, 0.08) (tri10, 0.01) (tri11, 0.02) (tri12, 0.21) (tri13, 0.01) (tri2, 0.29) (tri3, 0.04) (tri4, 0.07) (tri5, 0.02) (tri6, 1.78) (tri7, 1.81) (tri8, 0.06) (tri9, 0.01) (trit3, 0.05) (trit4, 0.49) (trit5, 207.84)};
      \addplot coordinates {(FK2, 1.70) (FK3, 0.80) (FK4, 0.87) (tri1, 0.98) (tri10, 0.85) (tri11, 0.88) (tri12, 1.04) (tri13, 0.83) (tri2, 1.16) (tri3, 0.90) (tri4, 0.94) (tri5, 0.87) (tri6, 1.27) (tri7, 1.45) (tri8, 0.91) (tri9, 0.82) (trit3, 0.87) (trit4, 1.04) (trit5, 5.21)};
    \end{axis}
  \end{tikzpicture}
\end{frame}

\begin{frame}
  \frametitle{Comparación}
  Paralelismo
  \noindent \begin{tikzpicture}
    \begin{axis}[
        ybar = 0.5pt, % Tipo de gráfico: barras verticales
        width = \textwidth,
        height = 0.8\textheight,
        bar width = 0.003\textwidth,
        symbolic x coords = {FK2, FK3, FK4, tri1, tri2, tri3, tri4, tri5, tri6, tri7, tri8, tri9, tri10, tri11, tri12, tri13, trit3, trit4, trit5}, % Etiquetas en el eje x
        xtick = data, % Mostrar marcas en el eje x para cada dato
        ylabel = {Tiempo en segundos}, % Etiqueta del eje y
        %xlabel = {Cantidad de hilos}, % Etiqueta del eje x
        xticklabel style = {rotate = 45}, % Rotar etiquetas
        ymin = 0, % Mínimo del eje y
        ymax = 215, % Máximo del eje y
        enlarge x limits = 0.03, % Espaciado en los bordes
        legend style = {at = {(0.024, 0.94)}, anchor = north west}, % Estilo de la leyenda
        legend entries = {1, 2, 4, 6, 8, 16}, % Entradas de la leyenda
    ]
      % Datos para cada conjunto
      \addplot coordinates {(FK2, 0.00) (FK3, 0.00) (FK4, 0.07) (tri1, 0.10) (tri10, 0.01) (tri11, 0.02) (tri12, 0.32) (tri13, 0.01) (tri2, 0.53) (tri3, 0.03) (tri4, 0.07) (tri5, 0.01) (tri6, 2.20) (tri7, 2.19) (tri8, 0.07) (tri9, 0.00) (trit3, 0.04) (trit4, 0.47) (trit5, 213.58)};
      \addplot coordinates {(FK2, 0.00) (FK3, 0.00) (FK4, 0.06) (tri1, 0.08) (tri10, 0.01) (tri11, 0.02) (tri12, 0.28) (tri13, 0.00) (tri2, 0.42) (tri3, 0.03) (tri4, 0.06) (tri5, 0.01) (tri6, 1.98) (tri7, 2.02) (tri8, 0.06) (tri9, 0.00) (trit3, 0.04) (trit4, 0.47) (trit5, 212.76)};
      \addplot coordinates {(FK2, 0.00) (FK3, 0.00) (FK4, 0.06) (tri1, 0.07) (tri10, 0.01) (tri11, 0.01) (tri12, 0.22) (tri13, 0.00) (tri2, 0.36) (tri3, 0.03) (tri4, 0.06) (tri5, 0.01) (tri6, 1.82) (tri7, 1.92) (tri8, 0.06) (tri9, 0.00) (trit3, 0.04) (trit4, 0.45) (trit5, 209.93)};
      \addplot coordinates {(FK2, 0.00) (FK3, 0.00) (FK4, 0.06) (tri1, 0.06) (tri10, 0.01) (tri11, 0.01) (tri12, 0.22) (tri13, 0.01) (tri2, 0.32) (tri3, 0.03) (tri4, 0.06) (tri5, 0.01) (tri6, 1.86) (tri7, 1.84) (tri8, 0.05) (tri9, 0.00) (trit3, 0.04) (trit4, 0.45) (trit5, 212.14)};
      \addplot coordinates {(FK2, 0.00) (FK3, 0.00) (FK4, 0.07) (tri1, 0.06) (tri10, 0.01) (tri11, 0.01) (tri12, 0.20) (tri13, 0.00) (tri2, 0.31) (tri3, 0.03) (tri4, 0.06) (tri5, 0.01) (tri6, 1.83) (tri7, 1.81) (tri8, 0.05) (tri9, 0.00) (trit3, 0.04) (trit4, 0.45) (trit5, 208.28)};
      \addplot coordinates {(FK2, 0.00) (FK3, 0.00) (FK4, 0.07) (tri1, 0.06) (tri10, 0.01) (tri11, 0.01) (tri12, 0.20) (tri13, 0.00) (tri2, 0.29) (tri3, 0.03) (tri4, 0.05) (tri5, 0.01) (tri6, 1.79) (tri7, 1.80) (tri8, 0.05) (tri9, 0.00) (trit3, 0.04) (trit4, 0.46) (trit5, 208.62)};
    \end{axis}
  \end{tikzpicture}
\end{frame}

\begin{frame}
  \frametitle{Comparación}
  Paralelismo sin trit5
  \noindent \begin{tikzpicture}
    \begin{axis}[
        ybar = 0.5pt, % Tipo de gráfico: barras verticales
        width = \textwidth,
        height = 0.8\textheight,
        bar width = 0.003\textwidth,
        symbolic x coords = {FK2, FK3, FK4, tri1, tri2, tri3, tri4, tri5, tri6, tri7, tri8, tri9, tri10, tri11, tri12, tri13, trit3, trit4}, % Etiquetas en el eje x
        xtick = data, % Mostrar marcas en el eje x para cada dato
        ylabel = {Tiempo en segundos}, % Etiqueta del eje y
        %xlabel = {Cantidad de hilos}, % Etiqueta del eje x
        xticklabel style = {rotate = 45}, % Rotar etiquetas
        ymin = 0, % Mínimo del eje y
        ymax = 2.5, % Máximo del eje y
        enlarge x limits = 0.03, % Espaciado en los bordes
        legend style = {at = {(0.024, 0.94)}, anchor = north west}, % Estilo de la leyenda
        legend entries = {1, 2, 4, 6, 8, 16}, % Entradas de la leyenda
    ]
      % Datos para cada conjunto
      \addplot coordinates {(FK2, 0.00) (FK3, 0.00) (FK4, 0.07) (tri1, 0.10) (tri10, 0.01) (tri11, 0.02) (tri12, 0.32) (tri13, 0.01) (tri2, 0.53) (tri3, 0.03) (tri4, 0.07) (tri5, 0.01) (tri6, 2.20) (tri7, 2.19) (tri8, 0.07) (tri9, 0.00) (trit3, 0.04) (trit4, 0.47)};
      \addplot coordinates {(FK2, 0.00) (FK3, 0.00) (FK4, 0.06) (tri1, 0.08) (tri10, 0.01) (tri11, 0.02) (tri12, 0.28) (tri13, 0.00) (tri2, 0.42) (tri3, 0.03) (tri4, 0.06) (tri5, 0.01) (tri6, 1.98) (tri7, 2.02) (tri8, 0.06) (tri9, 0.00) (trit3, 0.04) (trit4, 0.47)};
      \addplot coordinates {(FK2, 0.00) (FK3, 0.00) (FK4, 0.06) (tri1, 0.07) (tri10, 0.01) (tri11, 0.01) (tri12, 0.22) (tri13, 0.00) (tri2, 0.36) (tri3, 0.03) (tri4, 0.06) (tri5, 0.01) (tri6, 1.82) (tri7, 1.92) (tri8, 0.06) (tri9, 0.00) (trit3, 0.04) (trit4, 0.45)};
      \addplot coordinates {(FK2, 0.00) (FK3, 0.00) (FK4, 0.06) (tri1, 0.06) (tri10, 0.01) (tri11, 0.01) (tri12, 0.22) (tri13, 0.01) (tri2, 0.32) (tri3, 0.03) (tri4, 0.06) (tri5, 0.01) (tri6, 1.86) (tri7, 1.84) (tri8, 0.05) (tri9, 0.00) (trit3, 0.04) (trit4, 0.45)};
      \addplot coordinates {(FK2, 0.00) (FK3, 0.00) (FK4, 0.07) (tri1, 0.06) (tri10, 0.01) (tri11, 0.01) (tri12, 0.20) (tri13, 0.00) (tri2, 0.31) (tri3, 0.03) (tri4, 0.06) (tri5, 0.01) (tri6, 1.83) (tri7, 1.81) (tri8, 0.05) (tri9, 0.00) (trit3, 0.04) (trit4, 0.45)};
      \addplot coordinates {(FK2, 0.00) (FK3, 0.00) (FK4, 0.07) (tri1, 0.06) (tri10, 0.01) (tri11, 0.01) (tri12, 0.20) (tri13, 0.00) (tri2, 0.29) (tri3, 0.03) (tri4, 0.05) (tri5, 0.01) (tri6, 1.79) (tri7, 1.80) (tri8, 0.05) (tri9, 0.00) (trit3, 0.04) (trit4, 0.46)};
    \end{axis}
  \end{tikzpicture}
\end{frame}

% \begin{frame}
%   \frametitle{Repaso}
% \end{frame}

\begin{frame}
  \frametitle{Trabajos futuros}
  \begin{itemize}
    \item Implementar la optimización de Faugère-Lachartre en F4
    \item Y si es posible paralelizarla
  \end{itemize}
\end{frame}

\end{document}
