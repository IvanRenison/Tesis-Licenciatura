\documentclass{amsbook}
\usepackage{amsthm}
\usepackage{fontspec}
\usepackage{enumitem}
\usepackage[bookmarks]{hyperref}
\defaultfontfeatures{Renderer=Basic,Ligatures={TeX}}
\usepackage[math-style=ISO,bold-style=ISO]{unicode-math}
\setmathfont{Asana Math}

\hfuzz=1000pt % Elimina warnings de "Overfull \hbox"

% \usepackage{silence} % Para eliminar el warning "Non-breaking space (`~') should have been used."
% \WarningFilter{latex}{Non-breaking space} % Pero no anda 😕

\usepackage[margin=5mm]{geometry} % Adjust the margin values as desired

\usepackage{float}
\usepackage[ruled, vlined, linesnumbered]{algorithm2e}

\usepackage{polyglossia}
\setdefaultlanguage{spanish}

\newtheoremstyle{customstyle} % <name>
{} % <Space above>
{} % <Space below>
{} % <Body font>
{} % <Indent amount>
{\bfseries} % <Theorem head font>
{:} % <Punctuation after theorem head>
{.5em} % <Space after theorem head>
{} % <Theorem head spec (can be left empty, meaning `normal`)

\theoremstyle{customstyle}
\newtheorem{definition}{Definición}[section]
\newtheorem{theorem}{Teorema}[section]
\addto\captionsspanish{\renewcommand{\proofname}{\textbf{Demostración:}}}




\DeclareMathOperator{\sop}{sop}
\DeclareMathOperator{\lm}{lm}
\DeclareMathOperator{\lc}{lc}
\DeclareMathOperator{\lt}{lt}
\DeclareMathOperator{\tail}{tail}
\DeclareMathOperator{\amb}{amb}
\renewcommand{\S}{\text{S}}
\DeclareMathOperator{\B}{B}


\begin{document}
\fontsize{16pt}{19pt}\selectfont % Increase the font size


\section{Definiciones preliminares}

En esta sección se explican los conceptos matemáticos necesarios para entender el resto del trabajo.
En particular se explica …. % Completar

\subsection{Sistemas de re-escritura}

En esta sub-sección se explican algunas definiciones y teoremas básicos de sistemas de re-escritura.

Para toda está sección fijemos un conjunto $A$ y $→ ⊆ A^2$ una relación sobre $A$.

\begin{definition}[Formas normales]
Sean $a, b ∈ A$:

  $a$ está en forma normal $⇔ ∄x ∈ A : a → x$

  $b$ es forma normal de $a ⇔ a →^* b ∧ b$ está en forma normal

  $a$ tiene forma normal $⇔ ∃x ∈ A : x$ es forma normal de $a$

  $a ↓ b ⇔ ∃x ∈ A : a →^* x ∧ b →^* x$
\end{definition}

\begin{definition}\ % Según internet acá tendría que usar \\, pero eso da error

  $→$ es confluente $⇔ ∀x, y, z ∈ A : x →^* y ∧ x →^*z ⇒ y ↓ z$

  $→$ es Church-Rosser $⇔ ∀x, y ∈ A : x ↔^* y ⇔ x ↓ y$

  $→$ es normalizante $⇔ ∀x ∈ A : x$ tiene forma normal

  $→$ es terminante $⇔ ∄X ∈ A^ℕ : ∀i ∈ ℕ : X_i → X_{i + 1}$


\end{definition}

\begin{theorem}\label{thm:confluente ⇒ normalizante}
$→$ es confluente $⇒ →$ es normalizante
\end{theorem}

\begin{theorem}\label{thm:confluente ⇔ Church-Rosser}
  $→$ es confluente $⇔ →$ es Church-Rosser
\end{theorem}

\subsection{Álgebra libre}

Ahora si empezamos con el tema de la tesis en si.

\begin{definition}[Monomio libre]
Sea $X$ un alfabeto finito:

$⟨X⟩ = X^*$

a $⟨X⟩$ se lo llama monomios libres sobre $X$.
\end{definition}

Por ejemplo, si $X = \{a, b, c\}$ algunos monomios son:

$m_0 = abbcb$

$m_1 = bbc$

$m_2 = bcb$

$m_3 = ε$

El $ε$ de $m_3$ es la palabra vacía. Notar que $m_1 ≠ m_2$ ya que el la concatenación no es conmutativa.

A partir de ahora fijamos $X$ un alfabeto finito. En $⟨X⟩$ se define el producto como la concatenación:

\begin{definition}[Producto de monomios libre]\

  $· : ⟨X⟩^2 → ⟨X⟩$

  $v · w = vw$

Y a la estructura $(⟨X⟩, ·)$ se la llama monoide libre sobre $⟨X⟩$.
\end{definition}

Siguiendo con el ejemplo de antes, tenemos $m_1 · m_2 = bccbcb$.

\begin{definition}[Divisibilidad]
Sean $v, w ∈ ⟨X⟩$:

  $v | w ⇔ ∃a , b ∈ ⟨X⟩ : w = avb$

Y cuando $v | w$ se dice que $v$ divide a $w$.
\end{definition}

Y en el ejemplo de antes tenemos por ejemplo $m_1 | m_0$ ya que $m_0 = a m_1 b$.

Mas adelante va a ser necesario tener un orden entre los elementos de $⟨X⟩$, pero no es necesario fijar uno concreto, así que lo que vamos a hacer es definir las propiedades que tienen que tener un orden para servir y trabajar con un orden cualquiera que las cumpla.

\begin{definition}[Buen orden monomial]
Sea $≤$ un orden total sobre $⟨X⟩$:

  $≤$ es un buen orden monomial si y solo si:
\begin{enumerate}[label = (\alph*)]
\item $∀v, w, a, b ∈ ⟨X⟩ : v ≤ w ⇒ avb ≤ awb$

\item $∀S ⊆ ⟨X⟩ : S ≠ Ø ⇒ S$ tiene mínimo elemento con respecto a $≤$
\end{enumerate}
\end{definition}

Ahora un ejemplo de orden que cumple con esta definición:

\begin{definition}[Orden lexicográfico por grado]
Fijemos $X = \{x_1, …, x_n\}$ y un orden total entre $X$: $x_1 ≤ … ≤ x_n$, el cual se extiende (como es usual) de forma lexicográfica a $⟨X⟩$

El orden lexicográfico por grado se define así:

  $≤_{deglex} ⊆ ⟨X⟩^2$

  $a ≤_{deglex} b ⇔ |a| < |b| ∨ (|a| = |b| ∧ a ≤ b)$
\end{definition}

O sea, el orden lexicográfico por grado es orden primero por cardinalidad, que se llama grado también, y desempatar por orden lexicográfico, por ejemplo tenemos $bbc ≤_{deglex} ab$, $aabbc ≤_{deglex} abbcc$, $ab ≤_{deglex} abc$ y $ε ≤_{deglex} a$.

Este orden se puede probar fácilmente que es un bueno orden monomial.

A partir de ahora fijamos un buen orden monomial $≤$, y usaremos $<$, $≥$ y $>$ como se usan habitualmente.

Ahora una propiedad sobre el orden monomial que es consecuencia directa de la definición:

\begin{theorem}[$≤$ no tiene cadenas estrictamente decrecientes infinitas]\label{thm:< no cadenas dec inf}\

  $∄M ∈ ⟨X⟩^ℕ : ∀i ∈ ℕ : M_{i + 1} < M_i$

\end{theorem}

Ahora pasamos a hablar de sumar monomios entre si para tener polinomios no conmutativos.

\begin{definition}[Álgebra libre (asociativa)]
Sea $R$ un anillo conmutativo

Se define $R⟨X⟩$ la $R$-álgebra libre sobre $X$ como:

  $R⟨X⟩ = \{\sum_{i = 1}^n c_i w_i : c_1, …, c_n ∈ R, w_1, …, w_n ∈ ⟨X⟩\}$

La suma en $R⟨X⟩$ se define de la manera esperable.

El producto por escalares de define como:

  $c (\sum_{i = 1}^n c_i w_i) = (\sum_{i = 1}^n c c_i w_i)$

El producto entre elementos de $R⟨X⟩$ se define como:

  $(\sum_{i = 1}^n c_i w_i) · (\sum_{i = 1}^m c'_i w'_i) = \sum_{i = 1}^n \sum_{j = 1}^m c_i c'_j w_i w_j$

A los elementos de $R⟨X⟩$ se los llama polinomios no conmutativos.
% Esta definición creo que hay que mejorarla, pero no se como
\end{definition}

Algunos ejemplos de polinomios no conmutativos sobre $ℚ⟨\{a, b, c\}⟩$ son los siguientes:

$p_0 = a$

$p_1 = ab + cb$

$p_2 = 3 abb + 4 bcca - 2 acab$

Notar que $p_1 ≠ ab + bc$ ya que el producto es no conmutativo.

Sobre los polinomios no conmutativos se hacen las siguientes definiciones, que serán de utilidad:


\begin{definition}
Sean $R$ un anillo conmutativo, $p ∈ R⟨X⟩$, $c_1, …, c_n ∈ R$, $w_1, …, w_n, w ∈ ⟨X⟩$, $f = \sum_{i = 1}^n c_i w_i$ y $≤$ un buen orden monomial.

Se define:

  $f_w = \left\{\begin{array}{ll} w = w_i → c_i \\ \text{si no} → 0  \end{array} \right. $

  $\sop(f) = \{w_1, …, w_n\}$

  $\lm_≤(f) = \min_≤(\sop(f))$

  $\lc_≤(f) = f_{\lm(f)}$

  $\lt_≤(f) = \lc_≤(f) · \lm_≤(f)$

  $\tail_≤(f) = f - \lt_≤(f)$

  $f$ es mónico $⇔ \lc_≤(f) = 1$

A $\sop(f)$ se lo llama soporte de $f$, a $\lm_≤(f)$ se lo llama monomio principal de $f$, a $\lc_≤(f)$ se lo llama coeficiente principal de $f$ y a $\lt_≤(f)$ se lo llama término principal de $f$. Los nombres lm, lc, lt y tail vienen del inglés leading monomial, leading coefficient, leading term y tail respectivamente.

\end{definition}

\begin{theorem}
Sea $R$ un anillo conmutativo, entonces:

  $(R⟨X⟩, +, ·)$ es un anillo

\end{theorem}

Esto significa que aplican todas las definiciones y teoremas de anillo.
En particular vale la definición de ideal y algunas definiciones y teoremas relacionados, que como son muy importantes para este trabajo las repasaremos a continuación.

\begin{definition}[Ideal]\label{def:ideal}
Sean $R$ un anillo e $I, B ⊆ R$

  $I$ es un ideal de $R ⇔$:
\begin{enumerate}[label = (\alph*)]
\item $I ≠ Ø$

\item $∀a, b ∈ I : a + b ∈ I$

\item $∀a ∈ I, r, r' ∈ R : r a r' ∈ I$
\end{enumerate}

  $(B) = \{\sum_{i = 1}^n c_i b_i c_i' : n ∈ ℕ, b_1, …, b_n ∈ B, c_1, …, c_n, c_1', …, c_n' ∈ R\}$
\end{definition}

\begin{theorem}
Sean $R$ un anillo y $B ⊆ R$:

  $(B)$ es un ideal de $R$

\end{theorem}

Además a $(B)$ se lo llama el ideal generado por $B$.

\begin{definition}[Congruencia modulo un ideal]\label{def:congruencia mod ideal}
Sean $R$ un anillo e $I ⊆ R$, se define la siguiente relación:

  $≡_I ⊆ R^2$

  $a ≡_I b ⇔ a - b ∈ I$

\end{definition}

\begin{theorem}\label{thm:congruencia mod ideal es equivalencia}
Sean $R$ un anillo y $I ⊆ R$ un ideal:

  $≡_I$ es una relación de equivalencia
\end{theorem}

\begin{theorem}\label{thm:en ideal ⇔ congruente 0}
Sean $R$ un anillo, $I ⊆ R$ un ideal y $a ∈ R$:

  $a ∈ I ⇔ a ≡_I 0$
\end{theorem}


A partir de ahora fijamos un cuerpo $K$.

Una pregunta que surge es dado un conjunto finito $F ⊆ K⟨X⟩$ y un elemento $f ∈ K⟨X⟩$, ¿se puede calcular si $f ∈ (F)$?

La respuesta a esa pregunta es que no, porque el problema es indecible. Sin embargo, si existen algoritmos que cuando $f ∈ (F)$ termina y devuelve una prueba, pero que cuando $f ∉ (F)$ puede no terminar. Sobre esos algoritmos se trata este trabajo.

Mas adelante va a hacer falta comprar no solo monomios si no también polinomios, así que $≤$ se extiende a $K⟨X⟩$ así:

\begin{definition}[Extensión de orden monomial a polinomios]
Sean $f, g ∈  K⟨X⟩$:

  $f < g ⇔ (f = 0 ∧ g ≠ 0) ∨ \lm(f) < \lm(g) ∨ (\lm(f) = \lm(g) ∧ \tail(f) < \tail(g))$

\end{definition}
Es decir, el orden en los polinomios es orden lexicográfico con el polinomio visto como una lista de monomios, sin coeficientes ordenada de mayor a menor.

Por ejemplo, tenemos estas desigualdades en $K⟨X⟩$:

$a < ab$

$bcc + aab < bcc + c$

$ac < ac + aa$

Ahora algunos teoremas sobre el orden de los polinomios.

\begin{theorem}
$<$ como relación de $K⟨X⟩$ es un pre-orden.
\end{theorem}

\begin{theorem}[$<$ como relación de $K⟨X⟩$ no tiene cadenas estrictamente decrecientes infinitas]\label{thm:< en KX no cadenas dec inf}
  $∄P ∈ K⟨X⟩^ℕ : ∀i ∈ ℕ : P_{i + 1} < P_i$
\end{theorem}
\begin{proof}
Supongamos que existe tal $P$ y tomemos uno que minimize $\lm(P_1)$ (1)

(tomar este mínimo está bien por el teorema \ref{thm:< no cadenas dec inf}) % Mencionar el teorema sobre monomios

Notar que:

(2) $∀i ∈ ℕ : \lm(P_i) = \lm(P_1)$

  Esto ya que no puede ser $\lm(P_i) < \lm(P_1)$ porque entonces $P_i, P_{i + 1}, …$ sería una cadena estrictamente decreciente infinita que rompería (1).


(3) $\tail(P_1), \tail(P_2), …$ es una cadena estrictamente decreciente infinita.

  Esto por aplicar la definición del orden polinomial, por (2) y por lo que estaos suponiendo sobre $P$.

(4) $P_1 ≠ 0$

  Esto por que claramente $0$ es un mínimo.

Sin embargo, (3) contradice (1) ya que por (4) $\lm(\tail(P_1)) < \lm(P_1)$.

\end{proof}

\begin{definition}[Reducciones]
Sean $F ⊆ K⟨X⟩$, $g ∈ K⟨X⟩ - \{0\}$, $a, b ∈ ⟨X⟩$ y $f, f' ∈ K⟨X⟩$:

  $f →_{≤, a, g, b} f' ⇔ \lm_≤(agb) ∈ \sop(f) ∧ f' = f - \frac{f_{\lm_≤(agb)}}{\lc_≤(g)}agb$

  $→_{≤, g} = \bigcup_{a, b ∈ ⟨X⟩} →_{≤, a, g, b}$

  $→_{≤, F} = \bigcup_{g ∈ K⟨X⟩ - \{0\}} →_{≤, g}$
\end{definition}

\begin{theorem}[Las reducciones achican]\label{thm:→ achican}
Sean $F ⊆ K⟨X⟩$, $g ∈ K⟨X⟩ - \{0\}$, $a, b ∈ ⟨X⟩$ y $f, f' ∈ K⟨X⟩$:
\begin{enumerate}[label = (\alph*)]
\item $f →_{≤, a, g, b} f' ⇒ f' < f$

\item $f →_{≤, g} f' ⇒ f' < f$

\item $f →_{≤, F} f' ⇒ f' < f$
\end{enumerate}
\end{theorem}
\begin{proof}
Por la definición de las reducciones (a) $⇒$ (b) $⇒$ (c), por lo cual alcanza con probar (a).

Supongamos $f →_{≤, a, g, b} f'$ (1)

Esto implica por definición de reducciones:

  $\lm_≤(agb) ∈ \sop(f)$ (2)

  $f' = f - \frac{f_{\lm_≤(agb)}}{\lc_≤(g)}agb$ (3)

Escribamos $f = \sum_{i = 1}^n c_i m_i$ con $c_1, …, c_n ∈ K$, $m_1, …, m_n ∈ ⟨X⟩, m_1 > m_2 > … > m_n$ (4)

Sea $i$ tal que $m_i = \lm_≤(agb)$ (5), el cual existe por (2)

Notar también que $m_i = \lm_≤(\frac{f_{\lm_≤(agb)}}{\lc_≤(g)}agb)$ (6) porque es (5) multiplicado por un escalar

Si $i ≠ 0$:

  (6) significa que los términos $c_1 m_1, c_2 m_2, …, c_{i-1}, m_{i-1}$ son iguales en $f$ y en $f'$ y no hay nada mas en el medio, porque $f'$ es $f$ con cosas menores o iguales a $\lm_≤(\frac{f_{\lm_≤(agb)}}{\lc_≤(g)}agb)$ restadas.

  Además, como $m_i = \lm_≤(\frac{f_{\lm_≤(agb)}}{\lc_≤(g)}agb)$, tenemos que $f'_{m_i} = 0$, por ende, el término que sigue después de $m_{i-1}$ (si es que hay) es menor que $m_i$ y por ende $' < f$.

Si $i = 0$ aplica lo mismo pero directamente a $m_0$ sin tener en cuenta los $w$ anteriores.

\end{proof}

\begin{theorem}\label{thm:suma →↓}
Sean $F ⊆ K⟨X⟩, f, f_0, f_1 ∈ K⟨X⟩$, entonces:

  $f_0 →_{≤, F} f_1 ⇒ f_0 + f ↓_{≤, F} f_1 + f$

\end{theorem}
\begin{proof}
Supongamos el antecedente $f_0 →_{≤, F} f_1$ (1)

Sean $g ∈ F, a, b ∈ K⟨X⟩$ tales que $f_1 = f_0 - \frac{{f_0}_{\lm_≤(agb)}}{g_{\lm_≤(agb)}} agb$ (2), los cuales existen por el antecedente y por definición de $→_{≤, F}$

Dividamos en casos según si $\lm_≤(agb) ∈ \sop(f)$:

Caso $\lm_≤(agb) ∉ \sop(f)$:

  En este caso tenemos $f_0 + f →_{≤, F} f_1 + f$ por tomar $g$, $a$ y $b$, y por ende tenemos $f_0 + f ↓_{≤, F} f_1 + f$


Caso $\lm_≤(agb) ∈ \sop(f)$ (3):

  Subcaso $f_{\lm_≤(agb)} = -{f_0}_{\lm_≤(agb)}$ (4):

  Este es el caso en el que $\lm_≤(agb)$ se cancela en la suma $f_0 + f$

  Como además $\lm_≤(agb) ∉ \sop(f_1)$ por (1), tenemos:

    $f_1 + f →_{≤, F} f_1 + f - \frac{f_{\lm_≤(agb)}}{g_{\lm_≤(agb)}} agb$

  $⇒$ ((4))

    $f_1 + f →_{≤, F} f_1 + f + \frac{{f_0}_{\lm_≤(agb)}}{g_{\lm_≤(agb)}} agb$

  $⇒$ ((2))

    $f_1 + f →_{≤, F} f_0 + f$

  $⇒$

    $f_1 + f ↓_{≤, F} f_0 + f$

  Subcaso $f_{\lm_≤(agb)} ≠ -{f_0}_{\lm_≤(agb)}$:

  En este caso $\lm_≤(agb)$ no se cancela en la suma $f_0 + f$, así que podemos aplicar $→_{≤, F}$:

    $f_0 + f →_{≤, F} f_0 + f - \frac{(f_0 + f)_{\lm_≤(agb)}}{g_{\lm_≤(agb)}} agb$

  $⇒$

    $f_0 + f →_{≤, F} f_0 + f - \frac{{f_0}_{\lm_≤(agb)} + f_{\lm_≤(agb)}}{g_{\lm_≤(agb)}} agb$

  $⇒$

    $f_0 + f →_{≤, F} f_0 + f - \frac{{f_0}_{\lm_≤(agb)}}{g_{\lm_≤(agb)}} agb - \frac{f_{\lm_≤(agb)}}{g_{\lm_≤(agb)}} agb$

  $⇒$ ((2))

    $f_0 + f →_{≤, F} f_1 + f - \frac{f_{\lm_≤(agb)}}{g_{\lm_≤(agb)}} agb$ (5)

  Además por (3) y (4) también tenemos $\lm_≤(agb) ∈ \sop(f_1 + f)$, así que tenemos:

    $f_1 + f →_{≤, F} f_1 + f - \frac{(f_1 + f)_{\lm_≤(agb)}}{g_{\lm_≤(agb)}} agb$

  $⇒$

    $f_1 + f →_{≤, F} f_1 + f - \frac{{f_1}_{\lm_≤(agb)} + f_{\lm_≤(agb)}}{g_{\lm_≤(agb)}} agb$

  $⇒$ ($\lm_≤(agb) ∉ \sop(f_1)$ porque $f_0 →_{≤, F} f_1$ y definición de $→_{≤, F}$)

    $f_1 + f →_{≤, F} f_1 + f - \frac{0 + f_{\lm_≤(agb)}}{g_{\lm_≤(agb)}} agb$

  $⇒$

    $f_1 + f →_{≤, F} f_1 + f - \frac{f_{\lm_≤(agb)}}{g_{\lm_≤(agb)}} agb$ (6)

  Por (5) y (6) tenemos $f_0 + f ↓_{≤, F} f_1 + f$

\end{proof}

\begin{theorem}[Clausura reflexo transitiva de las reducciones]\label{thm:→^* = ≡}
Sea $F ⊆ K⟨X⟩$, entonces:

  $↔^*_{≤, F} = ≡_{(F)}$

(Donde $≡_{(F)}$ es la congruencia modulo un ideal definida en la definición \ref{def:congruencia mod ideal})
\end{theorem}
\begin{proof}
Lo vamos a probar por doble inclusión

Prueba de $↔^*_{≤, F} ⊆ ≡_{(F)}$:

  Como tanto $↔^*_{≤, F}$ como $≡_{(F)}$ son relaciones de equivalencia y además $↔^*_{≤, F}$ es la mínima relación de equivalencia que contiene a $→_{≤, F}$, así que alcanza con probar $→_{≤, F} ⊆ ≡_{(F)}$

  Supongamos $f →_{≤, F} f'$ (1), queremos probar $f ≡_{(F)} f'$

  Sean $g ∈ F, a, b ∈ ⟨X⟩$ tales que $f' = f - \frac{f_{\lm_≤(agb)}}{\lc_≤(g)}agb$, los cuales existen por (1) y la definición de $→_{≤, F}$

  Tenemos:

    $f ≡_{(F)} f'$

  $⇔$

    $f - f' ∈ (F)$

  $⇔$

    $f - (f - \frac{f_{\lm_≤(agb)}}{\lc_≤(g)}agb) ∈ (F)$

  $⇔$

    $\frac{f_{\lm_≤(agb)}}{\lc_≤(g)}agb ∈ (F)$

  Y esto último es claramente cierto por la definición de ideal (\ref{def:ideal})

\end{proof}

\begin{theorem}[Las reducciones se mantienen en ideal]\label{thm:→ mantiene pertenencia a ideal}
Sean $F ⊆ K⟨X⟩, f, g ∈ (F)$, entonces:

  $f →^*_{≤, F} f' ⇒ (f ∈ (F) ⇔ g ∈ (F))$

\end{theorem}
\begin{proof}
Si asumimos $f →^*_{≤, F} f'$ tenemos por el teorema \ref{thm:→^* = ≡} $f ≡_{(F)} f'$ y entonces por el teorema \ref{thm:en ideal ⇔ congruente 0} tenemos $f ∈ (F) ⇔ f' ∈ (F)$
\end{proof}

\begin{theorem}
Sea $F ⊆ K⟨X⟩$, entonces:

  $→_{≤, F}$ es normalizante

  $→_{≤, F}$ es terminante
\end{theorem}
\begin{proof}
Por el teorema \ref{thm:confluente ⇒ normalizante} que sea terminante implica que es normalizante, por lo cual con probar que es terminante alcanza.

Supongamos que $→_{≤, F}$ no es terminante, por definición de terminante podemos tomar una cadena $P ∈ K⟨X⟩^ℕ$ tal que:

$∀i ∈ ℕ : P_i →_{≤, F} P_{i+1}$

Por el teorema \ref{thm:→ achican} esto implica que:

$∀i ∈ ℕ : P_i > P_{i+1}$

Pero esto contradice el teorema \ref{thm:< en KX no cadenas dec inf} que dice que no hoy cadenas estrictamente decrecientes infinitas en $K⟨X⟩$.

\end{proof}

\begin{theorem}[Caracterización de las formas normales de $→$]
Sea $F ⊆ K⟨X⟩, f ∈ K⟨X⟩$, entonces:

  $f$ está en forma normal con respecto a $→_{≤, F} ⇔ ∄g ∈ F, m ∈ \sop(f) : \lm(g) | m$

\end{theorem}
\begin{proof}
Por contradicción, supongamos que tenemos $g ∈ F, m ∈ \sop(f)$ tal que $\lm(g) ∈ \sop(f)$

Sean $a, b ∈ ⟨X⟩$ tal que $m = agb$, los cuales existen por la definición de divisibilidad.

Entonces tenemos $f →_{≤, F} f - \frac{f_{\lm_≤(agb)}}{\lc_≤(g)}agb$ y por ende $f$ no está en forma normal.

\end{proof}

Sin embargo $→_{≤, F}$ no necesariamente es confluente, eso motiva las siguientes dos definiciones.

\begin{definition}[Bases de Gröbner]
Sean $I$ un ideal de $K⟨X⟩$ y $F ⊆ K⟨X⟩$:

  $F$ es una base de Gröbner de $I ⇔ (F) = I ∧ →_{≤, F}$ es confluente

  $F$ es una base de Gröbner reducida de $I ⇔ F$ es una base de Gröbner de $I ∧ ∀g ∈ F : g$ es irreducible con respecto a $→_{≤, F - \{g\}}$

\end{definition}

\begin{definition}[Estrategia de reducción]
Sea $e_≤ : 𝒫(K⟨X⟩) → K⟨X⟩ → K⟨X⟩$

Se define:

  $e_≤$ es una estrategia de reducción $⇔$

    $∀F ⊆ K⟨X⟩, f ∈ K⟨X⟩ : e_≤(F)(f)$ es forma normal de $f$ con respecto a $→_{≤, F}$
\end{definition}

En los casos en los que $F$ es una base de Gröbner y por ende $→_{≤, F}$ es confluente hay una sola $e_≤(F)$ posible, pero cuando no es confluente puede haber muchas. Tener definida la estrategia de reducción como una función nos va a venir bien mas adelante para hablar de una forma normal manteniendo la generalidad de como se calcula una forma normal de cada elemento.

Un ejemplo de estrategia de reducción podría calcularse con el siguiente seudocódigo:

\begin{algorithm}[H] % La H es para que se quede acá, porque se iba a otra página. Estaría bueno hacerlo global
  \caption{Ejemplo de estrategia de reducción}\label{alg:estrategia de reducción}
  \KwData{$F = \{f_1, …, f_n\} ⊆ K⟨X⟩, g ∈ K⟨X⟩$}
  \KwResult{$g' ∈ K⟨X⟩$}
  $g' ← g$

  $i ← 1$

  \While{$i ≤ n$} {
    \While{$i ≤ n$} {
      \If{$f_i ∈ \sop(g')$} {
        $g' ← g' - \frac{g'_{\lm(f_i)}}{\lc(f_i)}f_i$

        $i ← 1$

        \textbf{break}
      }
      \Else{
        $i ← i + 1$
      }
    }
  }
  \Return{$g'$}
\end{algorithm}

Este algoritmo consiste básicamente en siempre buscar entre los elementos de $G$ si hay alguno con el que reducir, y parar cuando ya no hay ninguno.

Una propiedad sobre las estrategias de reducción que vamos a necesitar es la siguiente:

\begin{theorem}[Las estrategias de reducción mantienen la pertenencia a ideales]\label{thm:e mantiene pertenencia a ideal}
Sean $e_≤$ una estrategia de reducción, $F ⊆ K⟨X⟩$ y $f ∈ (F)$, entonces:

  $e_≤(F)(f) ∈ (F)$

\end{theorem}
\begin{proof}
Como $e_≤(F)(f)$ es una forma normal de $f$ tenemos $f →^*_{≤, F} e_≤(F)(f)$ y por ende por el teorema \ref{thm:S es cerrado en ideal} y que $F ⊆ K⟨X⟩$ tenemos $e_≤(F)(f) ∈ (F)$.
\end{proof}


\begin{theorem}[Equivalencias de base de Gröbner]\label{thm:equivalencias de base de Gröbner}
Sean $I$ un ideal de $K⟨X⟩$, $G ⊆ K⟨X⟩$ y $e_≤$ una estrategia de reducción.

Son equivalentes:
\begin{enumerate}
\item $G$ es una base de Gröbner de $I$

\item $∀f ∈ K⟨X⟩ : (f ∈ I ⇔ f →^*_{≤, G} 0)$

\end{enumerate}
% Quizás agregue mas en algún momento

\end{theorem}
\begin{proof}\

Primero probemos (1) $⇒$ (2):

Supongamos que $G$ es una base de Gröbner de $I$ y tomemos $f ∈ K⟨X⟩$.

  Tenemos que probar $f ∈ I ⇔ f →^*_{≤, G} 0$

  Vamos de un lado para el otro:

    $f ∈ I$

  $⇔$ (Teorema \ref{thm:en ideal ⇔ congruente 0})

    $f ≡_I 0$

  $⇔$ (Teorema \ref{thm:→^* = ≡})

    $f ↔^*_{≤, G} 0$

  $⇔$ (Al ser $G$ una base de Gröbner $→_{≤, G}$ es confluente y por ende por el teorema \ref{thm:confluente ⇔ Church-Rosser} es Church-Rosser, así que aplico el sii de Church-Rosser)

    $f ↓_{≤, G} 0$

  $⇔$ (Definición de $↓$)

    $∃f' ∈ K⟨X⟩ : f →^*_{≤, G} f' ∧ 0 →^*_{≤, G} f'$

  $⇔$ (Como $0$ es el mínimo elemento y el teorema \ref{thm:→ achican} dice que las reducciones achican el segundo término del $∧$ ocurre solo para $f' = 0$)

    $f →^*_{≤, G} 0$

Ahora probemos (2) $⇒$ (1):

Supongamos $∀f ∈ K⟨X⟩ : (f ∈ I ⇔ f →^*_{≤, G} 0)$ (3), tenemos que probar que $G$ es una base de Gröbner de $I$:

  $G$ es una base de Gröbner de $I$

$⇔$ (Definición de base de Gröbner)

  $(G) = I ∧ →_{≤, G}$ es confluente

Probemos cada termino del $∧$ por separado:

Prueba de $(G) = I$:

  Tomemos $f ∈ K⟨X⟩$ y probemos $f ∈ (G) ⇔ f ∈ I$, probando ida y vuelta por separado:

  Ida ($⇒$):

    Supongamos $f ∈ (G)$

    Sean $c_1, …, c_n, c_1', …, c_n' ∈ K⟨X⟩$, $g_1, …, g_n ∈ G$ tales que $f = \sum_{i = 1}^n c_i g_i c_i'$, los cuales existen por la definición de $(\ ·\ )$

    Definamos $f_0 = f$ y para $i ∈ \{1, …, n\}$ $f_i = f_{i-1} - c_i g_i c_i'$

    Notar que tenemos $∀i ∈ \{1, …, n\} : f_{i-1} →_{≤, G} f_i$ y que $f_n = 0$

    Esto significa que $f →^*_{≤, G} 0$ y por ende por (3) $f ∈ I$

  Vuelta ($⇐$):

    Supongamos $f ∈ I$

    Por (3) tenemos que $f →^*_{≤, G} 0$

    Así que sean $f_0, f_1, …, f_n ∈ K⟨X⟩$ tales que $f_0 = f$, $f_n = 0$ y $∀i ∈ \{1, …, n\} : f_{i-1} →_{≤, G} f_i$, los cuales existen por la definición de $^*$

    Además, para cada $i ∈ \{1, …, n\}$ sean $c_i, c_i' ∈ K⟨X⟩, g_i ∈ G$ tales que $f_i = f_{i-1} - c_i g_i c_i'$, los cuales existen por definición de $→_{≤, G}$

    Notar que en particular $f_{i-1} = f_i + c_i g_i c_i'$ y por ende $f = \sum_{i = 1}^n c_i g_i c_i'$, lo cual prueba que $f ∈ (G)$

Prueba de $→_{≤, G}$ es confluente:

  Por definición de confluencia tenemos que $→_{≤, G}$ es confluente $⇔ ∀f, f_0, f_1 ∈ K⟨X⟩ : f →^*_{≤, G} f_0 ∧ f →^*_{≤, G} f_1 ⇒ f_0 ↓_{≤, G} f_1$

  Así que tomemos $f, f_0, f_1 ∈ K⟨X⟩$ y probemos el implica yendo de un lado para el otro

  $f →^*_{≤, G} f_0 ∧ f →^*_{≤, G} f_1$

$⇒$

  $f_0 ↔^*_{≤, G} f_1$

$⇒$ (Teorema \ref{thm:→^* = ≡})

  $f_0 ≡_{(G)} f_1$

$⇒$ (Definición $≡_{\ ·\ }$)

  $f_0 - f_1 ∈ (G)$

$⇒$ ((3), ya probamos que $(G) = I$)

  $f_0 - f_1 →^*_{≤, G} 0$

$⇒$ (Teorema \ref{thm:suma →↓})

  $(f_0 - f_1) + f_1 ↓_{≤, G} 0 + f_1$

$⇒$

  $f_0 ↓_{≤, G} f_1$

Con esto se termina la prueba.

\end{proof}


\subsection{Algoritmo de Buchberger}

\begin{definition}[Ambigüedades]
Sean $p, q, a, b, c, d ∈ ⟨X⟩$

  $(a, b, c, d, p, q)$ es una ambigüedad $ ⇔ apb = cqd ∧ |a|, |b| < |q| ∧ |c|, |d| < |p|$

La ambigüedad $(a, b, c, d, p, q)$ se dice que es:

  De superposición $⇔ a = ε = d ∨ b = ε = c$

  De inclusión $⇔ a = ε = b ∨ c = ε = d$

  Relevante $⇔$ es de superposición o de inclusión

Además, si $f, g ∈ K⟨X⟩$ se dice que $(a, b, c, d, f, g)$ es una ambigüedad si y solo si $(a, b, c, d, \lm_≤{(f)}, \lm_≤{(g)})$ es una ambigüedad y lo mismo para ambigüedades de superposición, de inclusión y relevantes.

Sea además $F ⊆ K⟨X⟩$:

  $\amb(f, g) = \{(a, b, c, d, f, g) : a, b, c, d ∈ ⟨X⟩ ∧ (a, b, c, d, f, g)\text{ es una ambigüedad}\}$

  $\amb(F) = \bigcup_{f, g ∈ F - \{0\}}{\amb(f, g)}$

\end{definition}

\begin{theorem}
Sean $a, b, c, d ∈ ⟨X⟩$ y $f, g ∈ K⟨X⟩$ entonces si $α = (a, b, c, d, f, g)$ es una ambigüedad:

  $\lm_≤{(afb)} = \lm_≤{(cgd)}$

\end{theorem}
\begin{proof}\

  $\lm_≤{(afb)}$

$=$ (Definición de ambigüedad para polinomios)

  $a\lm_≤{(f)}b$

$=$ (Definición de ambigüedad)

  $c\lm_≤{(g)}d$

$=$ (Definición de ambigüedad para polinomios)

  $\lm_≤{(cgd)}$

\end{proof}

Eso motiva la siguiente definición.

\begin{definition}[Monomios principales de ambigüedades]
Sean $a, b, c, d ∈ ⟨X⟩, f, g ∈ K⟨X⟩$ y $α = (a, b, c, d, f, g)$ una ambigüedad.

  $\lm_≤{(α)} = \lm_≤{(afb)}$
\end{definition}

Notar que también $\lm_≤{(α)} = \lm_≤{(cgb)}$

\begin{definition}[S-polinomios]
Sean $a, b, c, d ∈ ⟨X⟩, f, g ∈ K⟨X⟩$ y $α = (a, b, c, d, f, g)$ una ambigüedad.

  $\S(α) = \frac{afb}{\lc_≤{(f)}} - \frac{cgd}{\lc_≤{(g)}}$
\end{definition}

\begin{theorem}
Sean $a, b, c, d ∈ ⟨X⟩, f, g ∈ K⟨X⟩$ y $α = (a, b, c, d, f, g)$ una ambigüedad, entonces:

  $\lm_≤{(\S(α))} < \lm_≤{(α)}$

\end{theorem}
\begin{proof}
Esto es porque en la resta $\frac{afb}{\lc_≤{(f)}} - \frac{cgd}{\lc_≤{(g)}}$ los monomios principales se cancelan
\end{proof}

\begin{theorem}\label{thm:S es cerrado en ideal}
Sean $I ⊆ K⟨X⟩$ un ideal $a, b, c, d ∈ ⟨X⟩, f, g ∈ I$ y $α = (a, b, c, d, f, g)$ una ambigüedad

  $\S(α) ∈ I$

\end{theorem}
\begin{proof}
En la definición de $\S$ se ve claramente que es una combinación lineal de $f$ y $g$ con elementos de $K⟨X⟩$ (en particular con los elementos $\frac{a}{\lc_≤{(f)}}$, $b$, $\frac{c}{\lc_≤{(g)}}$ y $d$)
\end{proof}

\begin{theorem}\label{thm:equivalencias de base de Gröbner (con ambs)}
Sean $I$ un ideal de $K⟨X⟩$ y $G ⊆ K⟨X⟩$.

Son equivalentes:
\begin{enumerate}
\item $G$ es una base de Gröbner de $I$

\item $∀α ∈ \amb(G) : \S(α) →^*_{≤, G} 0$

\end{enumerate}
\end{theorem}
% TODO: Agregar prueba. De cualquier manera, Hof no lo prueba a esto

% Agregar algunos teoremas sobre S polinomios, ambigüedades, y bases de Gröbner

\begin{definition}[Conjuntos del algoritmo de Buchberger]
Sean $F ⊆ K⟨X⟩$ y $e_≤$ una estrategia de reducción:

  $\B_{e_≤}^0(F) = F$

  $\B_{e_≤}^{i + 1}(F) = \B_{e_≤}^i(F) ∪ \{e_≤(\B_{e_≤}^i(F))(\S(α)) : α ∈ \amb(\B_{e_≤}^i(F))\}$

  $\B_{e_≤}(F) = \bigcup_{i = 0}^∞ \B_{e_≤}^i(F)$

\end{definition}

\begin{theorem}
Sean $F ⊆ K⟨X⟩$ y $e_≤$ una estrategia de reducción, entonces:

  $\B_{e_≤}(F)$ es una base de Gröbner de $(F)$ (1)

  $(F)$ tiene una base de Gröbner finita $⇒ ∃i ∈ ℕ : (\B_{e_≤}^i(F))$ es una base de Gröbner (2)
\end{theorem}
\begin{proof}
Primero vamos a probar algunas afirmaciones intermedias:

(3) $∀i ∈ ℕ : \B_{e_≤}^{i}(F) ⊆ (F)$

  Por inducción en $i$ el caso base es valido porque $F ⊆ (F)$, para el caso inductivo supongamos que vale para $i$ y probemos que vale para $i + 1$:

  Tomemos $f ∈ \B_{e_≤}^{i + 1}(F)$ y probemos que $f ∈ (F)$.

  Por la definición recursiva de $\B_{e_≤}^{i + 1}$ tenemos $f ∈ \B_{e_≤}^i(F) ∨ ∃α ∈ \amb(\B_{e_≤}^i(F)) : f = e_≤(\B_{e_≤}^i(F))(\S(α))$

  El caso $f ∈ \B_{e_≤}^i(F)$ es valido por hipótesis inductiva.

  Para el otro caso tomemos ese $α$:

    Por el teorema \ref{thm:S es cerrado en ideal} tenemos que $\S(α) ∈ \B_{e_≤}^i(F)$ y por la hipótesis inductiva que $\S(α) ∈ (F)$.

    Esto implica por el teorema \ref{thm:e mantiene pertenencia a ideal} que $e_≤(\B_{e_≤}^i(F))(\S(α)) ∈ \B_{e_≤}^i(F)$ y por la hipótesis inductiva que $e_≤(\B_{e_≤}^i(F))(\S(α)) ∈ (F)$.

    Con lo cual queda probado (3).

(4) $∀i ∈ ℕ : \B_{e_≤}(F) ⊆ (F)$

  Esto consecuencia directa de (3) y la definición de $\B_{e_≤}$

(5) $(\B_{e_≤}(F)) = (F)$

  Esto vale porque por el caso base de la definición de $\B_{e_≤}^i$ tenemos $F ⊆ \B_{e_≤}$ y por (4).

(6) $∀α ∈ \amb(\B_{e_≤}(F)) : \S(α) →^*_{≤, \B_{e_≤}(F)} 0$

  Tomemos $α ∈ \amb(\B_{e_≤}(F))$ y probemos el $∀$

  Sea $(a, b, c, d, f, g) = α$

  Por definición de $\amb$ vale que $f, g ∈ \B_{e_≤}(F)$

  Sea $i ∈ ℕ$ el mínimo tal que $f ∈ \B_{e_≤}^i(F)$ y $j ∈ ℕ$ el mínimo tal que $g ∈ \B_{e_≤}^j(F)$

  Tomemos $k = \max(i, j)$

  Notar que $α ∈ \amb(\B_{e_≤}^k(F))$ por la definición de $\amb$ y por ende $\S(α) ∈ \B_{e_≤}(F)$

  Por definición de $→^*_{≤, F}$ esto significa que $\S(α) →^*_{≤, \B_{e_≤}(F)} 0$, así que queda probado (6).

Y ahora, por (5) y (6) vale la definición de base de Gröbner, así que queda probado (1).

% TODO: Probar (2)

\end{proof}


% \begin{algorithm}
%   \caption{Buchberger}
%   \KwData{Datos de entrada}
%   \KwResult{Resultado esperado}
%   Inicializar variables\;
%   \For{cada elemento en la lista}{
%     \If{condición}{
%       Realizar acción
%     }
%     \Else{
%       Realizar otra acción
%     }
%   }
% \end{algorithm}


\end{document}
