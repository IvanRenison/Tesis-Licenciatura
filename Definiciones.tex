\documentclass{amsbook}
\usepackage{fontspec}
\defaultfontfeatures{Renderer=Basic,Ligatures={TeX}}
\usepackage[math-style=ISO,bold-style=ISO]{unicode-math}
\setmathfont{Asana Math}

\usepackage[margin=5mm]{geometry} % Adjust the margin values as desired

\begin{document}
\fontsize{16pt}{19pt}\selectfont % Increase the font size

\section{Monomios}

Sea $X$ un alfabeto finito

Se define:

    $⟨X⟩ = X^*$

Se define el producto como la concatenación:

    $· : ⟨X⟩^2 → ⟨X⟩$

    $v · w = vw$

Y a $(⟨X⟩, ·)$ se lo llama monoide libre sobre $⟨X⟩$.

Sean $v, w ∈ ⟨X⟩$:

Se define:

    $v | w ⇔ ∃a , b ∈ ⟨X⟩ : w = avb$

Sea $≤$ un orden total sobre $⟨X⟩$

Se define:

    $≤$ es un buen orden monomial si y solo si:

    a) $∀v, w, a, b ∈ ⟨X⟩ : v ≤ w ⇒ avb ≤ awb$

    b) $∀S ⊆ ⟨X⟩ : S ≠ Ø ⇒ S$ tiene mínimo elemento con respecto a $≤$

\section{Álgebra libre (asociativa):}

Sean:

$R$ un anillo conmutativo

$X$ un alfabeto

Se define $R⟨X⟩$ la $R$-álgebra libre sobre $X$ como:

    $R⟨X⟩ = \{\sum_{i = 1}^n c_i w_i : c_1, …, c_n ∈ R, w_1, …, w_n ∈ ⟨X⟩\}$

La suma en $R⟨X⟩$ se define de la manera esperable.

El producto por escalares de define como:

    $c (\sum_{i = 1}^n c_i w_i) = (\sum_{i = 1}^n c c_i w_i)$

El producto entre elementos de $R⟨X⟩$ se define como:

    $(\sum_{i = 1}^n c_i w_i) · (\sum_{i = 1}^m c'_i w'_i) = \sum_{i = 1}^n \sum_{j = 1}^m c_i c'_j w_i w_j$

\section{Cosas de los polinomios no conmutativos:}

Sean:

$p ∈ R⟨X⟩$

$c_1, …, c_n ∈ R$

$w_1, …, w_n, w ∈ ⟨X⟩$

$f = \sum_{i = 1}^n c_i w_i$

$≤$ un buen orden monomial

Se define:

    $f_w = \left\{\begin{array}{ll} w = w_i → c_i \\ \text{si no} → 0  \end{array} \right. $

    $\text{sup}(f) = \{w_1, …, w_n\}$

    $\text{lm}_≤(f) = \min_≤(\text{sup}(f))$

    $\text{lc}_≤(f) = f_{\text{lm}(f)}$

    $\text{lt}_≤(f) = \text{lc}_≤(f) · \text{lm}_≤(f)$

    $f$ es mónico $⇔ \text{lc}_≤(f) = 1$

\section{Forman un anillo}

$(R⟨X⟩, +, ·)$ es un anillo, por lo tanto aplican todas las definiciones de anillo.

En particular vale la definición de ideal, que como es muy importante para este trabajo la escribo acá también:

Sean:

$R$ un anillo

$I ⊆ R$

$B ⊆ R$

Se define:

    $I$ es un ideal de $R ⇔$:

        a) $I ≠ Ø$

        b) $∀a, b ∈ I : a + b ∈ I$

        c) $∀a ∈ I, r, r' ∈ R : rar' ∈ I$

    $(B) = \{\sum_{i = 1}^n c_i b_i c_i' : n ∈ ℕ, b_1, …, b_n ∈ B, c_1, …, c_n, c_1', …, c_n' ∈ R\}$

Se puede probar que $(B)$ es un ideal de $R$ y a $(B)$ se lo llama el ideal generado por $B$.

\section{Reducciones}

Sean:

$X$ un alfabeto

$K$ un cuerpo

$G ⊆ K⟨X⟩$

$g ∈ K⟨X⟩ - \{0\}$

$a, b ∈ ⟨X⟩$

$f, f' ∈ K⟨X⟩$

$≤$ un buen orden monomial

Se define:

    $f →_{≤, a, g, b} f' ⇔ \text{lm}_≤(agb) ∈ \text{sup}(f) ∧ f' = f - \frac{f_{\text{lm}_≤(agb)}}{\text{lc}_≤(g)}agb$

    $→_{≤, g} = \bigcup_{a, b ∈ ⟨X⟩} →_{≤, a, g, b}$

    $→_{≤, G} = \bigcup_{g ∈ K⟨X⟩ - \{0\}} →_{≤, g}$

Se puede probar que $→_{≤, G}$ es terminante, y que hay casos en los que no es confluente. Eso motiva la siguiente definición.

\section{Bases de Gröbner}

Sean:

$X$ un alfabeto

$K$ un cuerpo

$I$ un ideal de $K⟨X⟩$

$G ⊆ K⟨X⟩$

$≤$ un buen orden monomial

Se define:

    $G$ es una base de Gröbner de $I ⇔ (G) = I ∧ →_{≤, G}$ es confluente


\end{document}
